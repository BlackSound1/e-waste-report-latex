\documentclass[10pt]{article}

\usepackage{amsmath}
\usepackage[english]{babel}
\usepackage{geometry}
\usepackage[utf8]{inputenc}
\usepackage[backend=biber, style=ieee]{biblatex}
\usepackage{csquotes}
\usepackage[indent=20pt]{parskip}
\usepackage{graphicx}
\usepackage{hyperref}


\geometry{a4paper, left=25mm, right=25mm, top=25mm, bottom=25mm}

\addbibresource{Project.bib}

\begin{document}
\title{How Can Canada Contribute to More Sustainable E-Waste Management Globally?}
\author{Matthew Segal}

\begin{titlepage}
\maketitle
\end{titlepage}

\tableofcontents

\newpage

\listoffigures

\newpage

\section{Introduction}

This is a research project examining the various sustainability aspects of electronic waste. 
I begin by defining the concept of e-waste and exploring basic facts about it, including its creation and eventual fate.
I then examine its inherent health, equity, and environmental impacts. Once the problems of e-waste are known, I show examples of what's being done about it globally and in Canada.
Finally, I propose solutions that can be implemented at the individual, Canadian, and international levels.  

In Appendix A, I examine Canada's place in the global e-waste system, with a focus on how developing countries are affected. 

As an extra analysis, I use Appendix B to examine the e-waste habits of Montrealers.
I explore two datasets and draw conclusions about the trends of how Montrealers dispose of their e-waste. 

\section{What is Electronic Waste?}

Electronic waste is a form of solid waste comprised of Electronic and Electrical Equipment (EEE) that is no longer used or is unwanted.
It is also called e-waste or Waste Electronic and Electrical Equipment (WEEE). I will use the terms e-waste and WEEE interchangeably.
E-waste comes from households, businesses, governments, and other institutions.
There are six different categories of e-waste defined in \cite[p. 19]{baldeGlobalEwasteMonitor2024}

\begin{enumerate}
    \item Temperature Exchange Equipment
    \item Screens and Monitors
    \item Lamps
    \item Large Equipment
    \item Small Equipment
    \item Small IT and Telecommunications Equipment
\end{enumerate}

These categories are based on the 54 UNU-KEYS defined in \cite[text]{fortiEWasteStatisticsGuidelines2018}, which are themselves derived
from the 58 UNU-KEYS defined in \cite{wangSystematicCompatibleClassification2012}. They encompass many kinds of electronic and
electrical equipment (EEE) seen daily, like fridges and televisions. They also include product categories that many people don't
often think about, like heat pumps and solar panels. 

It's also important to discuss certain things that do not count as e-waste. For instance, batteries are considered a separate waste stream \cite[p. 20]{baldeGlobalEwasteMonitor2024}.
, EEE that is only intended to function as part of a vehicle and not in isolation, like a car stereo system, is not e-waste \cite[p. 20]{baldeGlobalEwasteMonitor2024}.
I also differentiate WEEE, referring to EEE that is truly waste, from used-EEE. Used-EEE will be an important designation to consider when discussing the transboundary flows of this material.

E-waste is an interesting category of waste to study because it requires specific techniques and technologies to handle properly \cite[p. 1]{wangUnderstandingEnvironmentalPollutions2020}.
Failing in this, serious harm can come to human health and the environment, as will be discussed later.

\section{How is E-Waste Generated?}

Throughout this report, I will refer to e-waste \textit{generation} as not just the discarding of EEE items, but also the lack of desire to keep such items.
E-waste is unwanted, not just thrown out. When a person upgrades their smartphone to a newer model, their old phone becomes e-waste even if it's not discarded.
The user has no intention of using the device again since it has been replaced. 

Several main drivers of e-waste generation are noted in \cite{baldeGlobalEwasteMonitor2024}. One of these drivers is technological progress \cite[p. 10]{baldeGlobalEwasteMonitor2024}.
As technological products become more advanced and efficient, older devices become more obsolete. Even if an older device is not truly obsolete,
it may still be less desirable than a more modern competitor.

In addition, several product categories have short product lifecycles \cite[p. 10]{baldeGlobalEwasteMonitor2024}, \cite{huModelingSustainableProduct2009}.
Smartphones are an easy example of this, with new ones releasing so frequently \cite{brownChallengesManagingComponent2012}. Coupled with the above point, this paints a bleak picture of a conveyor
belt of new products for people to admire while old products fall off the other end as waste.

Electronics are becoming increasingly accessible to people all around the world \cite[p. 10]{baldeGlobalEwasteMonitor2024}. Rural areas are becoming more connected to the internet and more aspects
of life are becoming digitized. People use EEE to work, learn, and play more than ever. Thus, there is an increasing demand for EEE. The more people own devices like this,
the more WEEE is eventually generated.

There's also the fact that many devices are hard to repair \cite[p. 10]{baldeGlobalEwasteMonitor2024}. Many devices are glued and soldered together nowadays in such a way as to make
user-servicing difficult \cite{perzanowskiConsumerPerceptionsRight2021}. Even if a device is large enough for user-servicing, like a tractor, many companies don't want users to repair
their own products \cite{perzanowskiConsumerPerceptionsRight2021}.

Finally, the infrastructure to support proper e-waste management is not always present \cite[p. 10]{baldeGlobalEwasteMonitor2024}.
Countries in the Global South, as we will see, frequently recycle WEEE in a dangerous way because formal, well-regulated processes and infrastructures don't exist.
This causes health and ecological concerns, as I will talk about extensively.

\section{Formal and Informal E-Waste Recycling}

The main dangers of e-waste I will focus on come from its recycling. In discussing the recycling of e-waste, it is important to delineate between formal and informal practices.
These practices can be viewed as belonging to a spectrum \cite{guha-khasnobisFormalityInformality2006}, but I will consider these two extremes separately.

\subsection{Formal E-Waste Recycling}

Formal recycling occurs in better controlled and better regulated facilities \cite[p. 157]{ceballosFormalElectronicRecycling2016}. These facilities make use of more advanced
processes, techniques, and technologies to recycle e-waste in a less harmful way \cite[p. 290]{perkinsEWasteGlobalHazard2014}.
For these reasons, I consider this to be the more desirable of the two extremes.

A major problem of formal recycling is that these advanced, well-regulated facilities are expensive to set up and maintain, so they are usually found in developed
countries to the exclusion of developing countries \cite[p. 290]{perkinsEWasteGlobalHazard2014}.

A highly unfortunate statistic is that as of 2022, only around 22.3\% of global e-waste is estimated to be formally recycled \cite[pp. 10, 30]{baldeGlobalEwasteMonitor2024}.
This means that most e-waste is not formally recycled.

Another such statistic is that since 2010, the rate of e-waste generation is almost five times faster than formal recycling can keep up with \cite[p. 16]{baldeGlobalEwasteMonitor2024}.
Formal e-waste recycling practices and facilities are already difficult enough to set up, doing so faster may not be feasible. Also, poorer countries don't necessarily have
formal facilities at all. The capacity of formal facilities in the Global North is already too small, given this statistic. Formalizing processes in the Global South is an
additional struggle that makes the goal of safely recycling e-waste worldwide more daunting.

Just because material is recycled formally, doesn't mean that the process is fully safe and without problems. Canada can recycle electronics formally, but Montreal,
a Canadian city, demonstrates that there are still potential dangers in formal WEEE recycling contexts. In a survey of three Montreal WEEE recycling companies \cite{coteWorkerHealthFormal2021},
workers' employment status (i.e. hired from hiring agency or not) affected risk of injury from occupational and chemical hazards, affected their access to personal
protective equipment, and affected access to training \cite[p. 304]{coteWorkerHealthFormal2021}.
Workers were specifically concerned about inhaling dust, gas, and metal; unexpected arrivals of unusual and hazardous materials; a lack of medical follow-up after incidents;
and ambiguous cleaning practices \cite[p. 304]{coteWorkerHealthFormal2021}. The companies were seen as doing more for ecological compliance than safety compliance \cite[p. 303]{coteWorkerHealthFormal2021}.
Just because formal recycling is regulated and advanced, does not mean that there aren't improvements to be made.

\subsection{Informal E-Waste Recycling}

As the opposite of formal e-waste recycling, informal e-waste recycling is defined accordingly. Informal recycling generally has less regard for safety standards \cite[p. 291]{perkinsEWasteGlobalHazard2014}
and generally has negative impact on human health and the environment \cite[p. 38]{baldeGlobalEwasteMonitor2024}.

Some disturbing aspects of informal e-waste recycling include the presence of child labor in some cases, and the use of rudimentary techniques such as open-air
acid baths and grilling circuit boards to remove electronic components \cite[p. 22]{leungEnvironmentalContaminationElectronic2006}.

Unfortunately, even though it's reasonable to believe that the increased regulation formal recycling enjoys causes good health and environmental outcomes,
that can't be fully assumed. The safety of formal WEEE recycling is not as well-studied as the lack of safety of informal recycling \cite[p. 293]{coteWorkerHealthFormal2021}.
It may be the case that formal recycling is only \textit{better} than informal recycling, but still not good and should still be avoided.
Further research into the safety of formal WEEE recycling is required.

\subsection{Overview of the Formal-Informal Spectrum of Activities}

\begin{figure}[h]
    \centering
    \includegraphics[width=\textwidth]{images/Recycling.png}
    \caption[Formal vs. informal recycling]{The different practices associated with formal and informal e-waste activities \cite[Fig. 2]{abalansaElectronicWasteEnvironmental2021}}
\end{figure}

The above diagram showcases a basic overview of the differences between formal and informal e-waste recycling practices. From it, we can see how much simpler the formal process is.
First, WEEE is collected, then it's recycled in a controlled way, and finally the material is sold. One could think of this as the gold standard.

The informal process is much more complicated. Many informal e-waste workers are migrants from nearby \cite[p. 7]{abalansaElectronicWasteEnvironmental2021}.
Workers must frequently scavenge for parts \cite[p. 7]{abalansaElectronicWasteEnvironmental2021}. Once these items are obtained, they are sorted into working and non-working categories.
If the device is found to be working, it can be refurbished and sold. One may consider this as the silver standard. The truly undesirable aspects of informal
e-waste recycling emerge in the rightmost branch. If a scavenged item is found to be non-working, it is dismantled for parts and/ or valuable chemicals. 

This dismantling may involve open grilling of circuit boards to loosen components, visible in \cite{greenpeaceBreakingElectricalComponents2005}.
The child labor aspect of this task can be seen in \cite{greenpeaceBoyRecyclesEWaste2022}. Another possible dismantling activity involves open-air
acid baths to separate precious metals like silver, visible in \cite{greenpeaceBurningEwaste2005}.

As I will show, the informal process of e-waste recycling poses serious hazards to health and the environment.
Considering that these activities are frequently performed in less developed countries, this poses equity concerns as well.

\section{Where Does E-Waste Go?} \label{sec:where}

Tracking the movements of e-waste is difficult. The Basel Convention, an international agreement governing the transboundary
movements of toxic and hazardous materials \cite{unitednationsBaselConventionControl1989}, does not fully record e-waste movements \cite[p. 17]{baldeGlobalTransboundaryEwaste2022}.
Used-EEE is not necessarily the same as WEEE, so the Convention does not require the recording of its transboundary movements \cite[p. 19]{baldeGlobalTransboundaryEwaste2022}.
WEEE movements are sometimes illegal \cite{unitednationsofficeondrugsandcrimeunodcTransnationalOrganizedCrime2013} and thus tracking is intentionally avoided \cite[pp. 24, 36]{baldeGlobalTransboundaryEwaste2022}.

Also, even if trade routes of WEEE are revealed, quantities that pass along those routes are not necessarily revealed as well \cite[p. 19]{baldeGlobalTransboundaryEwaste2022}.
For instance, the UN Comtrade database records international trade statistics \cite{unitednationsComtrade}.
Canada has e-waste export statistics for 2022 and 2023, with nothing before, and each year containing different kinds of data \cite{unitednations}.

A confounding issue is that, even though researchers try to track e-waste, their works often have different geographical scopes and unharmonized data \cite[p. 19]{baldeGlobalTransboundaryEwaste2022}.

An attempt to compile known routes is found in \cite[p. 13]{baldeGlobalTransboundaryEwaste2022}. A simplified version of the diagram is found in \cite[p, 38]{baldeGlobalEwasteMonitor2024},
reproduced here:

\begin{figure}[h]
    \centering
    \includegraphics[width=\textwidth]{"images/E-Waste Flows.png"}
    \caption[Estimate E-Waste Flows in 2019]{Estimate of global transboundary e-waste flows in 2019 \cite[Fig. 16]{baldeGlobalEwasteMonitor2024}}
\end{figure}

Most important to this discussion are the pink lines shown above. These correspond to the uncontrolled movements of used-EEE and WEEE. As we can see,
much of this material leaves richer regions like North America, Europe, and Australia to enter poorer regions like South America, Southeast Asia, and Western Africa.

A problem with this map is that it is divided into broad regions, not countries. It's sometimes difficult to find country-specific export statistics
for this sort of material, as I show with Canada later. Interestingly, this map shows flows within regions, as well. 

\subsection{Informal E-Waste Hubs}

It is frequently the case that wealthier countries export their WEEE to developing countries \cite[p. 4]{wangUnderstandingEnvironmentalPollutions2020},
\cite[Sec. 4]{sthiannopkaoHandlingEwasteDeveloped2013}. This raises some immediate equity concerns, which will be discussed later.

This is a non-exhaustive map showing several cities around the world that serve as informal e-waste recycling hubs. These cities are adapted from \cite[Sec. 2.1]{abalansaElectronicWasteEnvironmental2021},
and include Renovación in Mexico City, Mexico; Santo André in São Paolo, Brazil; Agbogbloshie in Accra, Ghana; Lagos, Nigeria; and Guiyu, China. 

\begin{figure}[h]
    \centering
    \includegraphics[width=\textwidth]{"images/WEEE Hub Map 3.png"}
    \caption[E-Waste Hub Map]{A map of some informal e-waste hubs. Derived from \cite{abalansaElectronicWasteEnvironmental2021}}
\end{figure}

Agbogbloshie and Guiyu come up frequently in the literature. Guiyu is sometimes referred to as the “… electronic graveyard of the world”  \cite[p. 2]{wangUnderstandingEnvironmentalPollutions2020}.
Agbogbloshie has been called “… one of the most polluted slums in the world” \cite[p. 1]{daumMoreSustainableTrajectory2017}.

\subsubsection{How do E-Waste Hubs Emerge?}

Several factors have shown to be important in the emergence and continued existence of informal e-waste hubs around the world.

These factors are sometimes geographical and historical \cite[p. 43]{davisPollutionHavenHypothesis2019}.
For instance, a possible reason for the emergence of the Agbogbloshie informal e-waste hub is the fact that it is
close to the important Ghanan port city of Tema \cite[p. 2]{grantMappingInvisibleReal2012}, \cite{afrolnewsGhanaBoomDangerous2010},
where imported WEEE can be easily transported to Agbogbloshie.

One of the reasons that richer countries export their e-waste to poorer countries is due to some of these poorer
countries' fewer regulations and cheaper labor \cite[p. 43]{davisPollutionHavenHypothesis2019}.
However, this alone may not be a good enough indicator. Something that may be more important is a lack of sub-national regulation
\cite[p. 37]{davisPollutionHavenHypothesis2019}. For instance, the Agbogbloshie site may have emerged, in part, due to ambiguous local land ownership 
\cite[Sec. 3.3]{davisPollutionHavenHypothesis2019}, \cite[pp. 2, 4]{grantOutPlaceGlobal2006}.

Sometimes, communities experience wide-scale livelihood loss and denial of opportunity \cite[Sec. 3.2, 5.1]{davisPollutionHavenHypothesis2019}.
An example of this is found in Guiyu. This city historically had unreliable agriculture due to frequent flooding of local waterways.
In addition, they lost the economic advantage these waterways offered when China started expanding highway construction in the region \cite[p. 982]{zhangGuiyuNationwidePolicy2009}.

Also, once an e-waste hub emerges, it must have some advantage to continue existing. Some of these advantages include local expertise accumulation,
improved supply chains, and improved economies of scale \cite[Sec. 3.5]{davisPollutionHavenHypothesis2019}.
A good example of this is Lagos, Nigeria. The Nigerian government decided to capitalize on their supply of local and imported e-waste.
Many Nigerians became experts in e-waste recycling and vocational schools, governments, and companies started training in it \cite[pp. 93-94]{sullivanTrashTreasureGlobal2014}.

\section{Problems with E-Waste}

This section will enumerate some findings of the health, environmental, and equity effects of informal e-waste recycling,
the dominant form of e-waste recycling in the world \cite[pp. 10, 30]{baldeGlobalEwasteMonitor2024}.

\subsection{Health Problems}

Informally recycling e-waste can pose problems for human health. An example is found in \cite{xuDifferentialProteomicExpression2016}.
Elevated levels of cadmium (Cd) and lead (Pb) were found in the placentas of pregnant women who lived
near the Guiyu WEEE recycling site in China, compared to nearby Shantou \cite[Sec. 3.1]{xuDifferentialProteomicExpression2016}.
They found that the Guiyu women took longer to gestate their babies and the babies they gave birth to were smaller \cite[Sec. 3.1]{xuDifferentialProteomicExpression2016}.
They also note that this exposure causes significant changes in multiple proteins and that cadmium exposure, in particular, can slow fetal growth 
\cite[Sec. 5]{xuDifferentialProteomicExpression2016}.

For another Guiyu example, \cite{xuBloodConcentrationsLead2018} found that school-age children had high levels of mercury (Hg) in their blood.
The researchers note that exposure to lead and mercury in youth can cause damage to DNA, potentially increasing risk for cancers \cite[p. 1491]{xuBloodConcentrationsLead2018}.

In Vietnam, \cite{eguchiOccurrencePerchlorateThiocyanate2014} found high levels of perchlorate in the blood of people near a WEEE recycling facility,
in comparison with a rural site. Perchlorate can be used in explosive devices \cite[p. 29]{eguchiOccurrencePerchlorateThiocyanate2014}.
These high levels were not otherwise associated with dietary and lifestyle habits \cite[p. 33]{eguchiOccurrencePerchlorateThiocyanate2014}.

Sufficiently informal and unregulated practices put workers at increased risk of physical injury as well.
In Ghana, researchers studying the Agbogbloshie WEEE recycling site found that only 38\% of workers used gloves, goggles,
or other protective equipment \cite[Fig. 5]{adanuChallengesAdoptingSustainable2020}.
The same study found that only 23\% of surveyed workers “occasionally” used specialized cable stripping
machines instead of burning cables to extract the valuable materials within \cite[p. 4]{adanuChallengesAdoptingSustainable2020}.

E-waste contains many toxic and dangerous chemicals, according to a meta-analysis \cite{parvezHealthConsequencesExposure2021}.
Some of these chemicals include mercury, manganese (Mg), chromium (Cr), nickel (Ni),
polycyclic aromatic hydrocarbons, polybrominated diphenyl ethers, polychlorinated bisphenols, dechlorane plus, and many others
\cite[Sec. "Results"]{parvezHealthConsequencesExposure2021}.

\subsection{Environmental Problems}

Informally recycling WEEE can prove to be a serious environmental hazard. However,
it is sometimes viewed more as an economical concern, than an ecological one \cite[p. 133]{albuquerqueElectronicJunkBest2019}.
In this section, I will enumerate several findings from researchers.

Taking data from \cite[Tbl. 4]{dengAtmosphericLevelsCytotoxicity2006}, \cite[Sec. 6.7]{worldhealthorganizationregionalofficeforeuropeAirQualityGuidelines2000},
\cite[Tbl. 1]{varConcentrationTrendSeasonal2000}, and \cite[Tbl. 1]{kimChemicalCompositionFine2003},
I created a chart measuring the concentration of lead (Pb) in the air of Guiyu, China, in comparison to two other Asian
cities and World Health Organization estimates of non-urban areas in Europe.

\newpage

\begin{figure}[h]
    \centering
    \includegraphics[width=\textwidth]{"images/Lead Air.png"}
    \caption[Lead air concentration]{Lead air concentration in Guiyu vs. selected control areas \cite{dengAtmosphericLevelsCytotoxicity2006},
    \cite{worldhealthorganizationregionalofficeforeuropeAirQualityGuidelines2000},
    \cite{varConcentrationTrendSeasonal2000}, \cite{kimChemicalCompositionFine2003}}
\end{figure}

TSP refers to Total Suspended Particles (particles less than 30-60 micrometers) and
PM means Particulate Matter (particles smaller than 2.5 micrometers) \cite[p. 6945]{dengAtmosphericLevelsCytotoxicity2006}.
As one can see, both measures are far higher in Guiyu than in other Asian cities like Tokyo and Seoul.
They are also much higher than what the World Health Organization estimates the average to be for non-urban areas in Europe.
Please note that for \cite{kimChemicalCompositionFine2003}, I considered the “Fine” parameter in Table 1,
as that relates to the same PM measure of 2.5 micrometers \cite[p. 754]{kimChemicalCompositionFine2003}
that \cite{dengAtmosphericLevelsCytotoxicity2006} uses. Please note that \cite[Sec. 6.7]{worldhealthorganizationregionalofficeforeuropeAirQualityGuidelines2000}
lists their measurement as 0.15 micrograms per cubic meter, which is equal to 150 nanograms per cubic meter, as shown in the above chart.
Also note that the WHO did not specify if they considered TSP or PM measures
\cite[Sec. 6.7]{worldhealthorganizationregionalofficeforeuropeAirQualityGuidelines2000}.

In \cite{ouaboEcologicalRiskHuman2019}, researchers took soil samples from the Makea, Ngodi, and New Bell informal e-waste recycling sites in Douala, Cameroon.
They compared these to soil samples from a control area. They were measuring concentrations of the heavy metals lead (Pb), nickel (Ni), zinc (Zn),
cadmium (Cd), chromium (Cr), copper (Cu), and cobalt (Co).

\newpage

\begin{figure}[h]
    \centering
    \includegraphics[width=\textwidth]{"images/Soil Samples.png"}
    \caption[Soil Samples in Cameroon]{\cite[Fig. 2]{ouaboEcologicalRiskHuman2019}}
\end{figure}

The researchers found that the heavy metal concentrations in the three e-waste recycling sites were much higher
than in the control site. Of these, they noted that lead, nickel, copper, and cadmium posed the highest ecological risk.
This is particularly troubling for lead since it is much more concentrated in these samples than any other metal.
Chromium was considered lower, but still significant risk.  Zinc was considered the lowest risk, ecologically.
In all, all three sites were considered to pose very high ecological risk \cite[p. 7]{ouaboEcologicalRiskHuman2019}.

Researchers in \cite{sepulvedaReviewEnvironmentalFate2010} summarized the diverse ecological effects of informal
e-waste recycling in China and India. They produced the following diagram:

\newpage

\begin{figure}[h]
    \centering
    \includegraphics[width=\textwidth]{"images/Overview.png"}
    \caption[Ecological Problems Diagram]{Summary of ecological problems of informal e-waste recycling \cite[Fig. 1]{sepulvedaReviewEnvironmentalFate2010}}
\end{figure}

As one can see, informal e-waste recycling practices have diverse and wide-reaching environmental effects.
Each step in the informal process (dumping, dismantling, burning, and acid leaching) causes its own set
of environmental problems. Each of these effects are complexly related in a web.
To describe a few of these effects: leachates from bathing e-waste in acid can leak into the soil;
fumes from burning can be transported long distances; effluents from the byproducts of leaching
can get into water and the life-forms there; and coarse particles from shredding e-waste can enter
the soil and eventually the groundwater. All these effects can harm humans as well, as we breathe the
contaminated air; drink, bathe, and fish in the contaminated water; and eat contaminated animals and plants.

\subsection{Equity Problems}

As shown in \ref{sec:where}, WEEE is frequently sent from richer to poorer countries, sometimes illegally
\cite{unitednationsofficeondrugsandcrimeunodcTransnationalOrganizedCrime2013}. If we also consider that poorer countries are less likely
to have developed formal WEEE recycling practices and infrastructure \cite[p. 290]{perkinsEWasteGlobalHazard2014},
we can conclude that less developed countries take on a greater health and environmental burden
in processing e-waste than more developed countries. 

However, poorer countries can be active attractors of WEEE because it is a valuable income stream for the local economy 
\cite{davisPollutionHavenHypothesis2019}, \cite[Sec. 6]{sthiannopkaoHandlingEwasteDeveloped2013}.
Although some workers may owe their livelihoods to e-waste, and smaller economies may benefit from the informal industry, we can't
ignore the serious health and ecological problems these people are the primary victims of.

This poses an inherent contradiction. If the Global North were to stop exporting e-waste to the Global
South completely, this may cause wide-scale livelihood loss and denial of opportunity of the sort that
caused some informal e-waste recycling sites to emerge in the first place \cite[p. 43]{davisPollutionHavenHypothesis2019}.
But continuing to supply the Global South with such dangerous materials poses long-term ill-effects as well.
Maintaining e-waste hubs in the Global South for local economy and ensuring local health and environmental safety are at odds.

\section{What's Being Done?}

This section will examine what is being done about the problem of e-waste. It will focus on two scopes:
the international scope, and the Canadian scope. In Appendix B, an additional scope will be considered: the local, Montreal scope.

\subsection{Internationally}

Internationally, the main document one will discover in searching about e-waste is
The Basel Convention on the Control of Transboundary Movements of Hazardous Wastes and their Disposal,
colloquially known as the Basel Convention \cite{unitednationsBaselConventionControl1989}.
This is a UN agreement that was first adopted in 1989 and first came into force in 1992 \cite{PartiesBaselConvention}.
Canada was an early adopter of the Convention, signing the agreement in 1989 and ratifying it in 1992 \cite{PartiesBaselConvention}.
Notably, The United States signed the agreement in 1990, but never ratified it \cite{PartiesBaselConvention}.
The agreement outlines principles and standards for exporting and importing hazardous wastes.
However, e-waste is sometimes considered a loophole \cite{wangTakeResponsibilityElectronicwaste2016}.
As I mentioned in the introduction, used-EEE is a usefully different category than WEEE \cite[Sec. 1]{schluepWhereAreWEEE2012}.
This lets developed countries trade e-waste as used electronics, not hazardous and toxic waste, letting them skirt the Convention
\cite{wangTakeResponsibilityElectronicwaste2016}, \cite[p. 697]{ghoshWasteElectricalElectronic2016}.

Another potential criticism of the Basel Convention is that it is a top-down approach that is not as
efficient or effective as it could be because it does not consider the private sector and was not created
with enough scientific data \cite{awasthiCircularEconomyElectronic2019}.

An example of this is in India. Where, even though transboundary flows of WEEE are illegal,
it still happens in large volume and adds a valuable income stream to the local economy \cite[p. 697]{ghoshWasteElectricalElectronic2016}.
Another example is China. It signed the Basel Convention and banned the import of WEEE,
but material still finds its way in, smuggled via Vietnam \cite{ghoshWasteElectricalElectronic2016}, \cite[p. 15]{wangEWasteChinaCountry2013}.

In response to the Basel Convention, regional agreements were devised that extended the Convention and hoped to patch its oversights.
One such agreement is the Africa-centric Bamako Convention \cite{u.n.environmentBamakoConvention2017}.
The Basel Convention encourages regional agreements such as these \cite{u.n.environmentBamakoConvention2017},
\cite[Art. 11]{unitednationsBaselConventionControl1989}.

As mentioned in \ref{sec:where}, tracking the transboundary movements of e-waste poses a significant challenge.
This is an important area requiring improvement. If e-waste is to be better regulated worldwide,
it must be possible to map its flows (imports, exports, interregional, and intraregional), and in what quantity.

In the European Union, the Restriction of Hazardous Substances in Electrical and Electronic Equipment (RoHS)
first came into force in 2003. The second version came into force in 2011 \cite{RoHSDirectiveEuropean2022}.
This directive restricts the use of hazardous material in EEE and is concerned with the ecological recovery
of WEEE items \cite[Art. 1]{europeanunionDirective2011652011}. It requires that manufacturers ensure their
products do not contain more than the maximum allowable amount, by weight, of certain substances
\cite[Art. 4]{europeanunionDirective2011652011}. One problem with the RoHS is that EU member states
are able to decide upon penalties for non-conformity on their own \cite[Art. 23]{europeanunionDirective2011652011}.
Another problem is that the scope of the Directive is limited by the large number of product
categories that are exempt from it, like military equipment and implantable medical devices
\cite[Art. 2]{europeanunionDirective2011652011}.

\subsection{In Canada}

Canada exists in a formal WEEE-recycling context \cite[p. 157]{ceballosFormalElectronicRecycling2016}.
However, there exists no federal legislation concerning e-waste.
Instead, the individual provinces and territories decide their own rules \cite[p. 69]{baldeGlobalEwasteMonitor2024}.
Interestingly, the territory of Nunavut has no WEEE legislation at all \cite[p. 70]{baldeGlobalEwasteMonitor2024}.

The closest Nunavut has to this is a non-binding guideline on hazardous waste that only briefly
mentions one proper, non-battery, e-waste category: computer parts
\cite[Sec. 2.1]{governmentofnunavut-departmentofenvironmentEnvironmentalGuidelineGeneral2010}.
In terms of indigenous Canadian issues, over 80\% of Nunavut's population is Inuit \cite{governmentofcanadaCensusProfile20162017}.
An interesting avenue of further research is how the lack of WEEE regulation in Nunavut affects this population.

Of the provinces and territories that do have e-waste legislation, the Extended Producer Responsibility (EPR)
model is used as a basis for such laws \cite[p. 70]{baldeGlobalEwasteMonitor2024}.
The EPR model posits that it is producers (and by extension, manufacturers and first importers)
who bear the responsibility of managing their end-of-life products. Companies can pass this additional
cost on to consumers in the form of an environmental handling fee, charged at time of sale \cite{baldeGlobalEwasteMonitor2024},
\cite{environmentandclimatechangecanadaOverviewExtendedProducer2007}.

The EPR model can be divided into two subcategories: Individual Producer Responsibility (IPR)
and Collective Producer Responsibility (CPR) \cite[p. 1043]{atasuExtendedProducerResponsibility2012}.
In CPR, companies work together as a group to share the effort of responsibly recycling unwanted electronics.
A problem with this is that CPR forces sharing the costs associated with e-waste recycling, as well
\cite[p. 1043]{atasuExtendedProducerResponsibility2012}. This may discourage companies from
designing more recyclable products, as more recycling means more cost for them \cite[p. 1048]{atasuExtendedProducerResponsibility2012}.
The EPR model can also cause recycling to be emphasized over reuse \cite[p. 674]{dalhammarEnablingReuseExtended2021},
which I will later argue is a powerful way to reduce WEEE generation.

EPR has additional problems. One such problem is that it's difficult to define good targets for EPR effectiveness
\cite{huismanWhereDidWEEE2006}. Different materials, products, and processes all pose different logistical and
ecological problems, some of which are more important than others \cite[Tbl. 2]{huismanWhereDidWEEE2006}.
In some jurisdictions of Canada, such as Quebec, there are no penalties for enterprises failing
to disclose the effectiveness of their EPR efforts because such a mandate was repealed in 2023
\cite[Sec. 53.1 (6)]{quebecRegulationRespectingRecovery2011}. Also in Quebec, there are
no penalties for failing to set up drop-off centers as defined in Section 16, because such a penalty was also repealed in 2023 
\cite[Sec. 56.1 (2)]{quebecRegulationRespectingRecovery2011}. A third problem is that there may be no way to tell if
a Producer Responsibility Organization (PRO) (an organization set up to oversee the EPR efforts of its member company/ companies)
meets targets due to a lack of disclosure, as seen in Quebec with the EPRA \cite[p. 1036]{leclercInformalEWasteFlows2023a}.

Thankfully, some parts of Canada have landfill bans on e-waste, such as Newfoundland and Labrador
\cite{departmentofenvironmentandclimatechangepollutionpreventiondivisionLandfillBansSpecial2023}, Nova Scotia
\cite{departmentofenvironmentandclimatechangeMaterialsBannedDisposal2009}, Prince Edward Island
(see section “electronics considered for special disposal only”) \cite{princeedwardislandWasteWatch2016},
and Vancouver (see section “Banned product stewardship materials”) \cite{metrovancouverDisposalBanProgram}.

In Canada, there are a few organizations of note. One of them is Electronic Product Stewardship Canada (EPSC)
\cite{EPSCElectronicsProduct}. This is a non-profit industry consortium. They favor regulations that are
flexible and industry-led and they focus on making businesses more efficient \cite[p. 8]{xavierCircularEconomyEwaste2021}.
For a more governmental approach, there exists the Canadian Council of Ministers of the Environment (CCME)
\cite{CanadianCouncilMinisters}. This is an intergovernmental group of federal, provincial, and territorial ministers
that was founded in 2004. They favor the EPR model and try to focus on reduction of consumption \cite[p. 8]{xavierCircularEconomyEwaste2021}.
They created a set of principles called \textit{Canada-Wide Principles for Electronics Product Stewardship},
which tries to harmonize Canada-wide approaches and standards 
\cite{canadiancouncilofministersoftheenvironmentccmeCanadaWidePrinciplesElectronics2004}.

The third and most interesting of these organizations is the Electronic Products Recycling Association (EPRA)
\cite{EPRAARPEElectronic}. This is an industry-led not-for-profit group that operates regulated Canada-wide
e-waste recycling programs. They are a Producer Responsibility Organization that uses the CPR model as multiple
companies are members of it. There is no obligation for companies or individuals to use the EPRA
\cite{leclercExtendedProducerResponsibility2020}. The EPRA, despite regulating e-waste
recycling centers nationwide, has optional membership \cite[p.294]{coteWorkerHealthFormal2021}.

Because of the fact that Canada has no federal WEEE legislation, companies that do business in
more than one province or territory, may find e-waste compliance difficulties between them
\cite[p. 70]{baldeGlobalEwasteMonitor2024}.

As a whole, Canada puts no pressure on manufacturers to encourage more recyclable or repairable electronics
\cite[p. 70]{baldeGlobalEwasteMonitor2024}. This is a point I will bring up later. 

In 2022, Canada generated approximately 774 million kilograms of e-waste
and formally recycled only around 89 million kilograms of it \cite[Tbl. A2.4]{baldeGlobalEwasteMonitor2024}.
This is a rate of about 11.5\%. When compared to the global formal e-waste recycling average of around 22.3\%
\cite[Fig. 7]{baldeGlobalEwasteMonitor2024}, we see that Canada does about twice as poorly as the already
low estimated global average. Canada does not seem to be actively trying to improve this statistic either,
as it has no recycling target in place \cite[Tbl. A2.4]{baldeGlobalEwasteMonitor2024}. However, 
\cite{xavierCircularEconomyEwaste2021} notes some challenges Canada faces when it comes to managing e-waste:
its large surface area, population density, and WEEE generated per capita. Canada has a large amount of WEEE
generated per capita, but its population density is sparse over its huge surface area.
This poses serious logistical and cost difficulties of properly and uniformly managing e-waste in Canada
\cite[Sec. 3.1]{xavierCircularEconomyEwaste2021}.


\section{Solutions}

\subsection{Solutions for Individuals}

\subsubsection{The 3R Initiative}

\subsubsection{A Better Version of 3R}

\subsection{Solutions for Canada}

\subsection{Solutions for the World}

\section{Conclusion}

\appendix

\section{Canada and the Global E-Waste System}

\section{What's Being Done in Montreal?}

\subsection{Quebec Provincial Law}

\subsection{Private and Public E-Waste Recycling in Montreal}

\subsection{The E-Waste Habits of Montrealers}

\subsubsection{Computers}

\subsubsection{Printers}

\subsubsection{Televisions}

\subsubsection{Cell Phones}

\subsubsection{Landline Phones}

\subsubsection{Audiovisual Equipment}

\subsubsection{Batteries}

\newpage

\printbibliography[title={Bibliography}]

\end{document}

