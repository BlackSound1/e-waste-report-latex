\documentclass[10pt]{article}

\usepackage{amsmath}
\usepackage[english]{babel}
\usepackage{geometry}
\usepackage[utf8]{inputenc}
\usepackage[backend=biber, style=ieee]{biblatex}
\usepackage{csquotes}
\usepackage[indent=20pt]{parskip}
\usepackage{graphicx}


\geometry{a4paper, left=25mm, right=25mm, top=25mm, bottom=25mm}

\addbibresource{Project.bib}

\begin{document}
\title{How Can Canada Contribute to More Sustainable E-Waste Management Globally?}
\author{Matthew Segal}

\begin{titlepage}
\maketitle
\end{titlepage}

\tableofcontents

\newpage

\listoffigures

\newpage

\section{Introduction}

This is a research project examining the various sustainability aspects of electronic waste. 
I begin by defining the concept of e-waste and exploring basic facts about it, including its creation and eventual fate.
I then examine its inherent health, equity, and environmental impacts. Once the problems of e-waste are known, I show examples of what's being done about it globally and in Canada.
Finally, I propose solutions that can be implemented at the individual, Canadian, and international levels.  

In Appendix A, I examine Canada's place in the global e-waste system, with a focus on how developing countries are affected. 

As an extra analysis, I use Appendix B to examine the e-waste habits of Montrealers.
I explore two datasets and draw conclusions about the trends of how Montrealers dispose of their e-waste. 

\section{What is Electronic Waste?}

Electronic waste is a form of solid waste comprised of Electronic and Electrical Equipment (EEE) that is no longer used or is unwanted.
It is also called e-waste or Waste Electronic and Electrical Equipment (WEEE). I will use the terms e-waste and WEEE interchangeably.
E-waste comes from households, businesses, governments, and other institutions.
There are six different categories of e-waste defined in \cite[p. 19]{baldeGlobalEwasteMonitor2024}

\begin{enumerate}
    \item Temperature Exchange Equipment
    \item Screens and Monitors
    \item Lamps
    \item Large Equipment
    \item Small Equipment
    \item Small IT and Telecommunications Equipment
\end{enumerate}

These categories are based on the 54 UNU-KEYS defined in \cite[text]{fortiEWasteStatisticsGuidelines2018}, which are themselves derived
from the 58 UNU-KEYS defined in \cite{wangSystematicCompatibleClassification2012}. They encompass many kinds of electronic and
electrical equipment (EEE) seen daily, like fridges and televisions. They also include product categories that many people don't
often think about, like heat pumps and solar panels. 

It's also important to discuss certain things that do not count as e-waste. For instance, batteries are considered a separate waste stream \cite[p. 20]{baldeGlobalEwasteMonitor2024}.
, EEE that is only intended to function as part of a vehicle and not in isolation, like a car stereo system, is not e-waste \cite[p. 20]{baldeGlobalEwasteMonitor2024}.
I also differentiate WEEE, referring to EEE that is truly waste, from used-EEE. Used-EEE will be an important designation to consider when discussing the transboundary flows of this material.

E-waste is an interesting category of waste to study because it requires specific techniques and technologies to handle properly \cite[p. 1]{wangUnderstandingEnvironmentalPollutions2020}.
Failing in this, serious harm can come to human health and the environment, as will be discussed later.

\section{How is E-Waste Generated?}

Throughout this report, I will refer to e-waste \textit{generation} as not just the discarding of EEE items, but also the lack of desire to keep such items.
E-waste is unwanted, not just thrown out. When a person upgrades their smartphone to a newer model, their old phone becomes e-waste even if it's not discarded.
The user has no intention of using the device again since it has been replaced. 

Several main drivers of e-waste generation are noted in \cite{baldeGlobalEwasteMonitor2024}. One of these drivers is technological progress \cite[p. 10]{baldeGlobalEwasteMonitor2024}.
As technological products become more advanced and efficient, older devices become more obsolete. Even if an older device is not truly obsolete,
it may still be less desirable than a more modern competitor.

In addition, several product categories have short product lifecycles \cite[p. 10]{baldeGlobalEwasteMonitor2024}, \cite{huModelingSustainableProduct2009}.
Smartphones are an easy example of this, with new ones releasing so frequently \cite{brownChallengesManagingComponent2012}. Coupled with the above point, this paints a bleak picture of a conveyor
belt of new products for people to admire while old products fall off the other end as waste.

Electronics are becoming increasingly accessible to people all around the world \cite[p. 10]{baldeGlobalEwasteMonitor2024}. Rural areas are becoming more connected to the internet and more aspects
of life are becoming digitized. People use EEE to work, learn, and play more than ever. Thus, there is an increasing demand for EEE. The more people own devices like this,
the more WEEE is eventually generated.

There's also the fact that many devices are hard to repair \cite[p. 10]{baldeGlobalEwasteMonitor2024}. Many devices are glued and soldered together nowadays in such a way as to make
user-servicing difficult \cite{perzanowskiConsumerPerceptionsRight2021}. Even if a device is large enough for user-servicing, like a tractor, many companies don't want users to repair
their own products \cite{perzanowskiConsumerPerceptionsRight2021}.

Finally, the infrastructure to support proper e-waste management is not always present \cite[p. 10]{baldeGlobalEwasteMonitor2024}.
Countries in the Global South, as we will see, frequently recycle WEEE in a dangerous way because formal, well-regulated processes and infrastructures don't exist.
This causes health and ecological concerns, as I will talk about extensively.

\section{Formal and Informal E-Waste Recycling}

The main dangers of e-waste I will focus on come from its recycling. In discussing the recycling of e-waste, it is important to delineate between formal and informal practices.
These practices can be viewed as belonging to a spectrum \cite{guha-khasnobisFormalityInformality2006}, but I will consider these two extremes separately.

\subsection{Formal E-Waste Recycling}

Formal recycling occurs in better controlled and better regulated facilities \cite[p. 157]{ceballosFormalElectronicRecycling2016}. These facilities make use of more advanced
processes, techniques, and technologies to recycle e-waste in a less harmful way \cite[p. 290]{perkinsEWasteGlobalHazard2014}.
For these reasons, I consider this to be the more desirable of the two extremes.

A major problem of formal recycling is that these advanced, well-regulated facilities are expensive to set up and maintain, so they are usually found in developed
countries to the exclusion of developing countries \cite[p. 290]{perkinsEWasteGlobalHazard2014}.

A highly unfortunate statistic is that as of 2022, only around 22.3\% of global e-waste is estimated to be formally recycled \cite[pp. 10, 30]{baldeGlobalEwasteMonitor2024}.
This means that most e-waste is not formally recycled.

Another such statistic is that since 2010, the rate of e-waste generation is almost five times faster than formal recycling can keep up with \cite[p. 16]{baldeGlobalEwasteMonitor2024}.
Formal e-waste recycling practices and facilities are already difficult enough to set up, doing so faster may not be feasible. Also, poorer countries don't necessarily have
formal facilities at all. The capacity of formal facilities in the Global North is already too small, given this statistic. Formalizing processes in the Global South is an
additional struggle that makes the goal of safely recycling e-waste worldwide more daunting.

Just because material is recycled formally, doesn't mean that the process is fully safe and without problems. Canada can recycle electronics formally, but Montreal,
a Canadian city, demonstrates that there are still potential dangers in formal WEEE recycling contexts. In a survey of three Montreal WEEE recycling companies \cite{coteWorkerHealthFormal2021},
workers' employment status (i.e. hired from hiring agency or not) affected risk of injury from occupational and chemical hazards, affected their access to personal
protective equipment, and affected access to training \cite[p. 304]{coteWorkerHealthFormal2021}.
Workers were specifically concerned about inhaling dust, gas, and metal; unexpected arrivals of unusual and hazardous materials; a lack of medical follow-up after incidents;
and ambiguous cleaning practices \cite[p. 304]{coteWorkerHealthFormal2021}. The companies were seen as doing more for ecological compliance than safety compliance \cite[p. 303]{coteWorkerHealthFormal2021}.
Just because formal recycling is regulated and advanced, does not mean that there aren't improvements to be made.

\subsection{Informal E-Waste Recycling}

As the opposite of formal e-waste recycling, informal e-waste recycling is defined accordingly. Informal recycling generally has less regard for safety standards \cite[p. 291]{perkinsEWasteGlobalHazard2014}
and generally has negative impact on human health and the environment \cite[p. 38]{baldeGlobalEwasteMonitor2024}.

Some disturbing aspects of informal e-waste recycling include the presence of child labor in some cases, and the use of rudimentary techniques such as open-air
acid baths and grilling circuit boards to remove electronic components \cite[p. 22]{leungEnvironmentalContaminationElectronic2006}.

Unfortunately, even though it's reasonable to believe that the increased regulation formal recycling enjoys causes good health and environmental outcomes,
that can't be fully assumed. The safety of formal WEEE recycling is not as well-studied as the lack of safety of informal recycling \cite[p. 293]{coteWorkerHealthFormal2021}.
It may be the case that formal recycling is only \textit{better} than informal recycling, but still not good and should still be avoided.
Further research into the safety of formal WEEE recycling is required.

\subsection{Overview of the Formal-Informal Spectrum of Activities}

\begin{figure}[h]
    \centering
    \includegraphics[width=\textwidth]{images/Recycling.png}
    \caption[Formal vs. informal recycling]{The different practices associated with formal and informal e-waste activities \cite[Fig. 2]{abalansaElectronicWasteEnvironmental2021}}
\end{figure}

The above diagram showcases a basic overview of the differences between formal and informal e-waste recycling practices. From it, we can see how much simpler the formal process is.
First, WEEE is collected, then it's recycled in a controlled way, and finally the material is sold. One could think of this as the gold standard.

The informal process is much more complicated. Many informal e-waste workers are migrants from nearby \cite[p. 7]{abalansaElectronicWasteEnvironmental2021}.
Workers must frequently scavenge for parts \cite[p. 7]{abalansaElectronicWasteEnvironmental2021}. Once these items are obtained, they are sorted into working and non-working categories.
If the device is found to be working, it can be refurbished and sold. One may consider this as the silver standard. The truly undesirable aspects of informal
e-waste recycling emerge in the rightmost branch. If a scavenged item is found to be non-working, it is dismantled for parts and/ or valuable chemicals. 

This dismantling may involve open grilling of circuit boards to loosen components, visible in \cite{greenpeaceBreakingElectricalComponents2005}.
The child labor aspect of this task can be seen in \cite{greenpeaceBoyRecyclesEWaste2022}. Another possible dismantling activity involves open-air
acid baths to separate precious metals like silver, visible in \cite{greenpeaceBurningEwaste2005}.

As I will show, the informal process of e-waste recycling poses serious hazards to health and the environment.
Considering that these activities are frequently performed in less developed countries, this poses equity concerns as well.

\section{Where Does E-Waste Go?}

Tracking the movements of e-waste is difficult. The Basel Convention, an international agreement governing the transboundary
movements of toxic and hazardous materials \cite{unitednationsBaselConventionControl1989}, does not fully record e-waste movements \cite[p. 17]{baldeGlobalTransboundaryEwaste2022}.
Used-EEE is not necessarily the same as WEEE, so the Convention does not require the recording of its transboundary movements \cite[p. 19]{baldeGlobalTransboundaryEwaste2022}.
WEEE movements are sometimes illegal \cite{unitednationsofficeondrugsandcrimeunodcTransnationalOrganizedCrime2013} and thus tracking is intentionally avoided \cite[pp. 24, 36]{baldeGlobalTransboundaryEwaste2022}.

Also, even if trade routes of WEEE are revealed, quantities that pass along those routes are not necessarily revealed as well \cite[p. 19]{baldeGlobalTransboundaryEwaste2022}.
For instance, the UN Comtrade database records international trade statistics \cite{unitednationsComtrade}.
Canada has e-waste export statistics for 2022 and 2023, with nothing before, and each year containing different kinds of data \cite{unitednations}.

A confounding issue is that, even though researchers try to track e-waste, their works often have different geographical scopes and unharmonized data \cite[p. 19]{baldeGlobalTransboundaryEwaste2022}.

An attempt to compile known routes is found in \cite[p. 13]{baldeGlobalTransboundaryEwaste2022}. A simplified version of the diagram is found in \cite[p, 38]{baldeGlobalEwasteMonitor2024},
reproduced here:

\begin{figure}[h]
    \centering
    \includegraphics[width=\textwidth]{"images/E-Waste Flows.png"}
    \caption[Estimate E-Waste Flows in 2019]{Estimate of global transboundary e-waste flows in 2019 \cite[Fig. 16]{baldeGlobalEwasteMonitor2024}}
\end{figure}

Most important to this discussion are the pink lines shown above. These correspond to the uncontrolled movements of used-EEE and WEEE. As we can see,
much of this material leaves richer regions like North America, Europe, and Australia to enter poorer regions like South America, Southeast Asia, and Western Africa.

A problem with this map is that it is divided into broad regions, not countries. It's sometimes difficult to find country-specific export statistics
for this sort of material, as I show with Canada later. Interestingly, this map shows flows within regions, as well. 

\subsection{Informal E-Waste Hubs}

It is frequently the case that wealthier countries export their WEEE to developing countries \cite[p. 4]{wangUnderstandingEnvironmentalPollutions2020},
\cite[Sec. 4]{sthiannopkaoHandlingEwasteDeveloped2013}. This raises some immediate equity concerns, which will be discussed later.

This is a non-exhaustive map showing several cities around the world that serve as informal e-waste recycling hubs. These cities are adapted from \cite[Sec. 2.1]{abalansaElectronicWasteEnvironmental2021},
and include Renovación in Mexico City, Mexico; Santo André in São Paolo, Brazil; Agbogbloshie in Accra, Ghana; Lagos, Nigeria; and Guiyu, China. 

\begin{figure}[h]
    \centering
    \includegraphics[width=\textwidth]{"images/WEEE Hub Map 3.png"}
    \caption[E-Waste Hub Map]{A map of some informal e-waste hubs. Derived from \cite{abalansaElectronicWasteEnvironmental2021}}
\end{figure}

Agbogbloshie and Guiyu come up frequently in the literature. Guiyu is sometimes referred to as the “… electronic graveyard of the world”  \cite[p. 2]{wangUnderstandingEnvironmentalPollutions2020}.
Agbogbloshie has been called “… one of the most polluted slums in the world” \cite[p. 1]{daumMoreSustainableTrajectory2017}.

\subsubsection{How do E-Waste Hubs Emerge?}

Several factors have shown to be important in the emergence and continued existence of informal e-waste hubs around the world.

These factors are sometimes geographical and historical \cite[p. 43]{davisPollutionHavenHypothesis2019}.
For instance, a possible reason for the emergence of the Agbogbloshie informal e-waste hub is the fact that it is
close to the important Ghanan port city of Tema \cite[p. 2]{grantMappingInvisibleReal2012}, \cite{afrolnewsGhanaBoomDangerous2010},
where imported WEEE can be easily transported to Agbogbloshie.

One of the reasons that richer countries export their e-waste to poorer countries is due to some of these poorer
countries' fewer regulations and cheaper labor \cite[p. 43]{davisPollutionHavenHypothesis2019}.
However, this alone may not be a good enough indicator. Something that may be more important is a lack of sub-national regulation
\cite[p. 37]{davisPollutionHavenHypothesis2019}. For instance, the Agbogbloshie site may have emerged, in part, due to ambiguous local land ownership 
\cite[Sec. 3.3]{davisPollutionHavenHypothesis2019}, \cite[pp. 2, 4]{grantOutPlaceGlobal2006}.

Sometimes, communities experience wide-scale livelihood loss and denial of opportunity \cite[Sec. 3.2, 5.1]{davisPollutionHavenHypothesis2019}.
An example of this is found in Guiyu. This city historically had unreliable agriculture due to frequent flooding of local waterways.
In addition, they lost the economic advantage these waterways offered when China started expanding highway construction in the region \cite[p. 982]{zhangGuiyuNationwidePolicy2009}.

Also, once an e-waste hub emerges, it must have some advantage to continue existing. Some of these advantages include local expertise accumulation,
improved supply chains, and improved economies of scale \cite[Sec. 3.5]{davisPollutionHavenHypothesis2019}.
A good example of this is Lagos, Nigeria. The Nigerian government decided to capitalize on their supply of local and imported e-waste.
Many Nigerians became experts in e-waste recycling and vocational schools, governments, and companies started training in it \cite[pp. 93-94]{sullivanTrashTreasureGlobal2014}.

\section{Problems with E-Waste}

\subsection{Health Problems}

\subsection{Environmental Problems}

\subsection{Equity Problems}

\section{What's Being Done?}

\subsection{Internationally}

\subsection{In Canada}

\section{Solutions}

\subsection{Solutions for Individuals}

\subsubsection{The 3R Initiative}

\subsubsection{A Better Version of 3R}

\subsection{Solutions for Canada}

\subsection{Solutions for the World}

\section{Conclusion}

\appendix

\section{Canada and the Global E-Waste System}

\section{What's Being Done in Montreal?}

\subsection{Quebec Provincial Law}

\subsection{Private and Public E-Waste Recycling in Montreal}

\subsection{The E-Waste Habits of Montrealers}

\subsubsection{Computers}

\subsubsection{Printers}

\subsubsection{Televisions}

\subsubsection{Cell Phones}

\subsubsection{Landline Phones}

\subsubsection{Audiovisual Equipment}

\subsubsection{Batteries}

\newpage

\printbibliography[title={Bibliography}]

\end{document}

