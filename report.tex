\documentclass[10pt]{article}

\usepackage{amsmath}
\usepackage[english]{babel}
\usepackage{geometry}
\usepackage[utf8]{inputenc}
\usepackage[backend=biber, style=ieee]{biblatex}
\usepackage{csquotes}
\usepackage[indent=20pt]{parskip}


\geometry{a4paper, left=25mm, right=25mm, top=25mm, bottom=25mm}

\addbibresource{Project.bib}

\begin{document}
\title{How Can Canada Contribute to More Sustainable E-Waste Management Globally?}
\author{Matthew Segal}

\begin{titlepage}
\maketitle
\end{titlepage}

\tableofcontents

\newpage

\section{Introduction}

This is a research project examining the various sustainability aspects of electronic waste. 
I begin by defining the concept of e-waste and exploring basic facts about it, including its creation and eventual fate.
I then examine its inherent health, equity, and environmental impacts. Once the problems of e-waste are known, I show examples of what's being done about it globally and in Canada.
Finally, I propose solutions that can be implemented at the individual, Canadian, and international levels. \par 

In Appendix A, I examine Canada's place in the global e-waste system, with a focus on how developing countries are affected.\par 

As an extra analysis, I use Appendix B to examine the e-waste habits of Montrealers.
I explore two datasets and draw conclusions about the trends of how Montrealers dispose of their e-waste. \par

\section{What is Electronic Waste?}

Electronic waste is a form of solid waste comprised of Electronic and Electrical Equipment (EEE) that is no longer used or is unwanted.
It is also called e-waste or Waste Electronic and Electrical Equipment (WEEE). I will use the terms e-waste and WEEE interchangeably.
E-waste comes from households, businesses, governments, and other institutions.
There are six different categories of e-waste defined in \cite[p. 19]{baldeGlobalEwasteMonitor2024}

\begin{enumerate}
    \item Temperature Exchange Equipment
    \item Screens and Monitors
    \item Lamps
    \item Large Equipment
    \item Small Equipment
    \item Small IT and Telecommunications Equipment
\end{enumerate}

These categories are based on the 54 UNU-KEYS defined in \cite[text]{fortiEWasteStatisticsGuidelines2018}, which are themselves derived
from the 58 UNU-KEYS defined in \cite{wangSystematicCompatibleClassification2012}. They encompass many kinds of electronic and
electrical equipment (EEE) seen daily, like fridges and televisions. They also include product categories that many people don't
often think about, like heat pumps and solar panels. \par

It's also important to discuss certain things that do not count as e-waste. For instance, batteries are considered a separate waste stream \cite[p. 20]{baldeGlobalEwasteMonitor2024}.
, EEE that is only intended to function as part of a vehicle and not in isolation, like a car stereo system, is not e-waste \cite[p. 20]{baldeGlobalEwasteMonitor2024}.
I also differentiate WEEE, referring to EEE that is truly waste, from used-EEE. Used-EEE will be an important designation to consider when discussing the transboundary flows of this material.\par

E-waste is an interesting category of waste to study because it requires specific techniques and technologies to handle properly \cite[p. 1]{wangUnderstandingEnvironmentalPollutions2020}.
Failing in this, serious harm can come to human health and the environment, as will be discussed later.

\section{How is E-Waste Generated?}

Throughout this report, I will refer to e-waste \textit{generation} as not just the discarding of EEE items, but also the lack of desire to keep such items.
E-waste is unwanted, not just thrown out. When a person upgrades their smartphone to a newer model, their old phone becomes e-waste even if it's not discarded.
The user has no intention of using the device again since it has been replaced. \par

Several main drivers of e-waste generation are noted in \cite{baldeGlobalEwasteMonitor2024}. One of these drivers is technological progress \cite[p. 10]{baldeGlobalEwasteMonitor2024}.
As technological products become more advanced and efficient, older devices become more obsolete. Even if an older device is not truly obsolete,
it may still be less desirable than a more modern competitor.\par

In addition, several product categories have short product lifecycles \cite[p. 10]{baldeGlobalEwasteMonitor2024}, \cite{huModelingSustainableProduct2009}.
Smartphones are an easy example of this, with new ones releasing so frequently \cite{brownChallengesManagingComponent2012}. Coupled with the above point, this paints a bleak picture of a conveyor
belt of new products for people to admire while old products fall off the other end as waste.\par

Electronics are becoming increasingly accessible to people all around the world \cite[p. 10]{baldeGlobalEwasteMonitor2024}. Rural areas are becoming more connected to the internet and more aspects
of life are becoming digitized. People use EEE to work, learn, and play more than ever. Thus, there is an increasing demand for EEE. The more people own devices like this,
the more WEEE is eventually generated.\par

There's also the fact that many devices are hard to repair \cite[p. 10]{baldeGlobalEwasteMonitor2024}. Many devices are glued and soldered together nowadays in such a way as to make
user-servicing difficult \cite{perzanowskiConsumerPerceptionsRight2021}. Even if a device is large enough for user-servicing, like a tractor, many companies don't want users to repair
their own products \cite{perzanowskiConsumerPerceptionsRight2021}.\par

Finally, the infrastructure to support proper e-waste management is not always present \cite[p. 10]{baldeGlobalEwasteMonitor2024}.
Countries in the Global South, as we will see, frequently recycle WEEE in a dangerous way because formal, well-regulated processes and infrastructures don't exist.
This causes health and ecological concerns, as I will talk about extensively.

\section{Formal and Informal E-Waste Recycling}

The main dangers of e-waste I will focus on come from its recycling. In discussing the recycling of e-waste, it is important to delineate between formal and informal practices.
These practices can be viewed as belonging to a spectrum 

\subsection{Formal E-Waste Recycling}

\subsection{Informal E-Waste Recycling}

\subsection{Overview of the Formal-Informal Spectrum of Activities}

\section{Where Does E-Waste Go?}

\subsection{Informal E-Waste Hubs}

\subsubsection{How do E-Waste Hubs Emerge?}

\section{Problems with E-Waste}

\subsection{Health Problems}

\subsection{Environmental Problems}

\subsection{Equity Problems}

\section{What's Being Done?}

\subsection{Internationally}

\subsection{In Canada}

\section{Solutions}

\subsection{Solutions for Individuals}

\subsubsection{The 3R Initiative}

\subsubsection{A Better Version of 3R}

\subsection{Solutions for Canada}

\subsection{Solutions for the World}

\section{Conclusion}

\appendix

\section{Canada and the Global E-Waste System}

\section{What's Being Done in Montreal?}

\subsection{Quebec Provincial Law}

\subsection{Private and Public E-Waste Recycling in Montreal}

\subsection{The E-Waste Habits of Montrealers}

\subsubsection{Computers}

\subsubsection{Printers}

\subsubsection{Televisions}

\subsubsection{Cell Phones}

\subsubsection{Landline Phones}

\subsubsection{Audiovisual Equipment}

\subsubsection{Batteries}

\newpage

\printbibliography[title={Bibliography}]

\end{document}

