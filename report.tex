\documentclass[10pt]{article}

\usepackage{amsmath}
\usepackage[english]{babel}
\usepackage{geometry}
\usepackage[utf8]{inputenc}
\usepackage[backend=biber, style=ieee]{biblatex}
\usepackage{csquotes}
\usepackage[indent=20pt]{parskip}
\usepackage{graphicx}
\usepackage{hyperref}
\usepackage{wrapfig}


\geometry{a4paper, left=25mm, right=25mm, top=25mm, bottom=25mm}

\addbibresource{Project.bib}

\begin{document}
\title{How Can Canada Contribute to More Sustainable E-Waste Management Globally?}
\author{Matthew Segal}

\begin{titlepage}
\maketitle
\end{titlepage}

\pagenumbering{roman}
\tableofcontents

\newpage

\listoffigures

\newpage

\pagenumbering{arabic}
\section{Introduction}

This is a research project examining the various sustainability aspects of electronic waste. 
I begin by defining the concept of e-waste and exploring basic facts about it, including its creation and eventual fate.
I then examine its inherent health, equity, and environmental impacts. Once the problems of e-waste are known, I show examples of what's being done about it globally and in Canada.
Finally, I propose solutions that can be implemented at the individual, Canadian, and international levels.  

In Appendix A, I examine Canada's place in the global e-waste system, with a focus on how developing countries are affected. 

As an extra analysis, I use Appendix B to examine the e-waste habits of Montrealers.
I explore two datasets and draw conclusions about the trends of how Montrealers dispose of their e-waste. 

\section{What is Electronic Waste?}

Electronic waste is a form of solid waste comprised of Electronic and Electrical Equipment (EEE) that is no longer used or is unwanted.
It is also called e-waste or Waste Electronic and Electrical Equipment (WEEE). I will use the terms e-waste and WEEE interchangeably.
E-waste comes from households, businesses, governments, and other institutions.
There are six different categories of e-waste defined in \cite[p. 19]{baldeGlobalEwasteMonitor2024}

\begin{enumerate}
    \item Temperature Exchange Equipment
    \item Screens and Monitors
    \item Lamps
    \item Large Equipment
    \item Small Equipment
    \item Small IT and Telecommunications Equipment
\end{enumerate}

These categories are based on the 54 UNU-KEYS defined in \cite[text]{fortiEWasteStatisticsGuidelines2018}, which are themselves derived
from the 58 UNU-KEYS defined in \cite{wangSystematicCompatibleClassification2012}. They encompass many kinds of electronic and
electrical equipment (EEE) seen daily, like fridges and televisions. They also include product categories that many people don't
often think about, like heat pumps and solar panels. 

It's also important to discuss certain things that do not count as e-waste. For instance, batteries are considered a separate waste stream \cite[p. 20]{baldeGlobalEwasteMonitor2024}.
, EEE that is only intended to function as part of a vehicle and not in isolation, like a car stereo system, is not e-waste \cite[p. 20]{baldeGlobalEwasteMonitor2024}.
I also differentiate WEEE, referring to EEE that is truly waste, from used-EEE. Used-EEE will be an important designation to consider when discussing the transboundary flows of this material.

E-waste is an interesting category of waste to study because it requires specific techniques and technologies to handle properly \cite[p. 1]{wangUnderstandingEnvironmentalPollutions2020}.
Failing in this, serious harm can come to human health and the environment, as will be discussed later.

\section{How is E-Waste Generated?}

Throughout this report, I will refer to e-waste \textit{generation} as not just the discarding of EEE items, but also the lack of desire to keep such items.
E-waste is unwanted, not just thrown out. When a person upgrades their smartphone to a newer model, their old phone becomes e-waste even if it's not discarded.
The user has no intention of using the device again since it has been replaced. 

Several main drivers of e-waste generation are noted in \cite{baldeGlobalEwasteMonitor2024}. One of these drivers is technological progress \cite[p. 10]{baldeGlobalEwasteMonitor2024}.
As technological products become more advanced and efficient, older devices become more obsolete. Even if an older device is not truly obsolete,
it may still be less desirable than a more modern competitor.

In addition, several product categories have short product lifecycles \cite[p. 10]{baldeGlobalEwasteMonitor2024}, \cite{huModelingSustainableProduct2009}.
Smartphones are an easy example of this, with new ones releasing so frequently \cite{brownChallengesManagingComponent2012}. Coupled with the above point, this paints a bleak picture of a conveyor
belt of new products for people to admire while old products fall off the other end as waste.

Electronics are becoming increasingly accessible to people all around the world \cite[p. 10]{baldeGlobalEwasteMonitor2024}. Rural areas are becoming more connected to the internet and more aspects
of life are becoming digitized. People use EEE to work, learn, and play more than ever. Thus, there is an increasing demand for EEE. The more people own devices like this,
the more WEEE is eventually generated.

There's also the fact that many devices are hard to repair \cite[p. 10]{baldeGlobalEwasteMonitor2024}. Many devices are glued and soldered together nowadays in such a way as to make
user-servicing difficult \cite{perzanowskiConsumerPerceptionsRight2021}. Even if a device is large enough for user-servicing, like a tractor, many companies don't want users to repair
their own products \cite{perzanowskiConsumerPerceptionsRight2021}.

Finally, the infrastructure to support proper e-waste management is not always present \cite[p. 10]{baldeGlobalEwasteMonitor2024}.
Countries in the Global South, as we will see, frequently recycle WEEE in a dangerous way because formal, well-regulated processes and infrastructures don't exist.
This causes health and ecological concerns, as I will talk about extensively.

\section{Formal and Informal E-Waste Recycling}

The main dangers of e-waste I will focus on come from its recycling. In discussing the recycling of e-waste, it is important to delineate between formal and informal practices.
These practices can be viewed as belonging to a spectrum \cite{guha-khasnobisFormalityInformality2006}, but I will consider these two extremes separately.

\subsection{Formal E-Waste Recycling}

Formal recycling occurs in better controlled and better regulated facilities \cite[p. 157]{ceballosFormalElectronicRecycling2016}. These facilities make use of more advanced
processes, techniques, and technologies to recycle e-waste in a less harmful way \cite[p. 290]{perkinsEWasteGlobalHazard2014}.
For these reasons, I consider this to be the more desirable of the two extremes.

A major problem of formal recycling is that these advanced, well-regulated facilities are expensive to set up and maintain, so they are usually found in developed
countries to the exclusion of developing countries \cite[p. 290]{perkinsEWasteGlobalHazard2014}.

A highly unfortunate statistic is that as of 2022, only around 22.3\% of global e-waste is estimated to be formally recycled \cite[pp. 10, 30]{baldeGlobalEwasteMonitor2024}.
This means that most e-waste is not formally recycled.

Another such statistic is that since 2010, the rate of e-waste generation is almost five times faster than formal recycling can keep up with \cite[p. 16]{baldeGlobalEwasteMonitor2024}.
Formal e-waste recycling practices and facilities are already difficult enough to set up, doing so faster may not be feasible. Also, poorer countries don't necessarily have
formal facilities at all. The capacity of formal facilities in the Global North is already too small, given this statistic. Formalizing processes in the Global South is an
additional struggle that makes the goal of safely recycling e-waste worldwide more daunting.

Just because material is recycled formally, doesn't mean that the process is fully safe and without problems. Canada can recycle electronics formally, but Montreal,
a Canadian city, demonstrates that there are still potential dangers in formal WEEE recycling contexts. In a survey of three Montreal WEEE recycling companies \cite{coteWorkerHealthFormal2021},
workers' employment status (i.e. hired from hiring agency or not) affected risk of injury from occupational and chemical hazards, affected their access to personal
protective equipment, and affected access to training \cite[p. 304]{coteWorkerHealthFormal2021}.
Workers were specifically concerned about inhaling dust, gas, and metal; unexpected arrivals of unusual and hazardous materials; a lack of medical follow-up after incidents;
and ambiguous cleaning practices \cite[p. 304]{coteWorkerHealthFormal2021}. The companies were seen as doing more for ecological compliance than safety compliance \cite[p. 303]{coteWorkerHealthFormal2021}.
Just because formal recycling is regulated and advanced, does not mean that there aren't improvements to be made.

\subsection{Informal E-Waste Recycling}

As the opposite of formal e-waste recycling, informal e-waste recycling is defined accordingly. Informal recycling generally has less regard for safety standards \cite[p. 291]{perkinsEWasteGlobalHazard2014}
and generally has negative impact on human health and the environment \cite[p. 38]{baldeGlobalEwasteMonitor2024}.

Some disturbing aspects of informal e-waste recycling include the presence of child labor in some cases, and the use of rudimentary techniques such as open-air
acid baths and grilling circuit boards to remove electronic components \cite[p. 22]{leungEnvironmentalContaminationElectronic2006}.

Unfortunately, even though it's reasonable to believe that the increased regulation formal recycling enjoys causes good health and environmental outcomes,
that can't be fully assumed. The safety of formal WEEE recycling is not as well-studied as the lack of safety of informal recycling \cite[p. 293]{coteWorkerHealthFormal2021}.
It may be the case that formal recycling is only \textit{better} than informal recycling, but still not good and should still be avoided.
Further research into the safety of formal WEEE recycling is required.

\subsection{Overview of the Formal-Informal Spectrum of Activities}

\begin{figure}[h]
    \centering
    \includegraphics[width=\textwidth]{images/Recycling.png}
    \caption[Formal vs. informal recycling]{The different practices associated with formal and informal e-waste activities \cite[Fig. 2]{abalansaElectronicWasteEnvironmental2021}}
\end{figure}

The above diagram showcases a basic overview of the differences between formal and informal e-waste recycling practices. From it, we can see how much simpler the formal process is.
First, WEEE is collected, then it's recycled in a controlled way, and finally the material is sold. One could think of this as the gold standard.

The informal process is much more complicated. Many informal e-waste workers are migrants from nearby \cite[p. 7]{abalansaElectronicWasteEnvironmental2021}.
Workers must frequently scavenge for parts \cite[p. 7]{abalansaElectronicWasteEnvironmental2021}. Once these items are obtained, they are sorted into working and non-working categories.
If the device is found to be working, it can be refurbished and sold. One may consider this as the silver standard. The truly undesirable aspects of informal
e-waste recycling emerge in the rightmost branch. If a scavenged item is found to be non-working, it is dismantled for parts and/ or valuable chemicals. 

This dismantling may involve open grilling of circuit boards to loosen components, visible in \cite{greenpeaceBreakingElectricalComponents2005}.
The child labor aspect of this task can be seen in \cite{greenpeaceBoyRecyclesEWaste2022}. Another possible dismantling activity involves open-air
acid baths to separate precious metals like silver, visible in \cite{greenpeaceBurningEwaste2005}.

As I will show, the informal process of e-waste recycling poses serious hazards to health and the environment.
Considering that these activities are frequently performed in less developed countries, this poses equity concerns as well.

\section{Where Does E-Waste Go?} \label{sec:where}

Tracking the movements of e-waste is difficult. The Basel Convention, an international agreement governing the transboundary
movements of toxic and hazardous materials \cite{unitednationsBaselConventionControl1989}, does not fully record e-waste movements \cite[p. 17]{baldeGlobalTransboundaryEwaste2022}.
Used-EEE is not necessarily the same as WEEE, so the Convention does not require the recording of its transboundary movements \cite[p. 19]{baldeGlobalTransboundaryEwaste2022}.
WEEE movements are sometimes illegal \cite{unitednationsofficeondrugsandcrimeunodcTransnationalOrganizedCrime2013} and thus tracking is intentionally avoided \cite[pp. 24, 36]{baldeGlobalTransboundaryEwaste2022}.

Also, even if trade routes of WEEE are revealed, quantities that pass along those routes are not necessarily revealed as well \cite[p. 19]{baldeGlobalTransboundaryEwaste2022}.
For instance, the UN Comtrade database records international trade statistics \cite{unitednationsComtrade}.
Canada has e-waste export statistics for 2022 and 2023, with nothing before, and each year containing different kinds of data \cite{unitednations}.

A confounding issue is that, even though researchers try to track e-waste, their works often have different geographical scopes and unharmonized data \cite[p. 19]{baldeGlobalTransboundaryEwaste2022}.

An attempt to compile known routes is found in \cite[p. 13]{baldeGlobalTransboundaryEwaste2022}. A simplified version of the diagram is found in \cite[p, 38]{baldeGlobalEwasteMonitor2024},
reproduced here:

\begin{figure}[h]
    \centering
    \includegraphics[width=\textwidth]{"images/E-Waste Flows.png"}
    \caption[Estimate E-Waste Flows in 2019]{Estimate of global transboundary e-waste flows in 2019 \cite[Fig. 16]{baldeGlobalEwasteMonitor2024}}
\end{figure}

Most important to this discussion are the pink lines shown above. These correspond to the uncontrolled movements of used-EEE and WEEE. As we can see,
much of this material leaves richer regions like North America, Europe, and Australia to enter poorer regions like South America, Southeast Asia, and Western Africa.

A problem with this map is that it is divided into broad regions, not countries. It's sometimes difficult to find country-specific export statistics
for this sort of material, as I show with Canada later. Interestingly, this map shows flows within regions, as well. 

\subsection{Informal E-Waste Hubs}

It is frequently the case that wealthier countries export their WEEE to developing countries \cite[p. 4]{wangUnderstandingEnvironmentalPollutions2020},
\cite[Sec. 4]{sthiannopkaoHandlingEwasteDeveloped2013}. This raises some immediate equity concerns, which will be discussed later.

This is a non-exhaustive map showing several cities around the world that serve as informal e-waste recycling hubs. These cities are adapted from \cite[Sec. 2.1]{abalansaElectronicWasteEnvironmental2021},
and include Renovación in Mexico City, Mexico; Santo André in São Paolo, Brazil; Agbogbloshie in Accra, Ghana; Lagos, Nigeria; and Guiyu, China. 

\begin{figure}[h]
    \centering
    \includegraphics[width=\textwidth]{"images/WEEE Hub Map 3.png"}
    \caption[E-Waste Hub Map]{A map of some informal e-waste hubs. Derived from \cite{abalansaElectronicWasteEnvironmental2021}}
\end{figure}

Agbogbloshie and Guiyu come up frequently in the literature. Guiyu is sometimes referred to as the “… electronic graveyard of the world”  \cite[p. 2]{wangUnderstandingEnvironmentalPollutions2020}.
Agbogbloshie has been called “… one of the most polluted slums in the world” \cite[p. 1]{daumMoreSustainableTrajectory2017}.

\subsubsection{How do E-Waste Hubs Emerge?}

Several factors have shown to be important in the emergence and continued existence of informal e-waste hubs around the world.

These factors are sometimes geographical and historical \cite[p. 43]{davisPollutionHavenHypothesis2019}.
For instance, a possible reason for the emergence of the Agbogbloshie informal e-waste hub is the fact that it is
close to the important Ghanan port city of Tema \cite[p. 2]{grantMappingInvisibleReal2012}, \cite{afrolnewsGhanaBoomDangerous2010},
where imported WEEE can be easily transported to Agbogbloshie.

One of the reasons that richer countries export their e-waste to poorer countries is due to some of these poorer
countries' fewer regulations and cheaper labor \cite[p. 43]{davisPollutionHavenHypothesis2019}.
However, this alone may not be a good enough indicator. Something that may be more important is a lack of sub-national regulation
\cite[p. 37]{davisPollutionHavenHypothesis2019}. For instance, the Agbogbloshie site may have emerged, in part, due to ambiguous local land ownership 
\cite[Sec. 3.3]{davisPollutionHavenHypothesis2019}, \cite[pp. 2, 4]{grantOutPlaceGlobal2006}.

Sometimes, communities experience wide-scale livelihood loss and denial of opportunity \cite[Sec. 3.2, 5.1]{davisPollutionHavenHypothesis2019}.
An example of this is found in Guiyu. This city historically had unreliable agriculture due to frequent flooding of local waterways.
In addition, they lost the economic advantage these waterways offered when China started expanding highway construction in the region \cite[p. 982]{zhangGuiyuNationwidePolicy2009}.

Also, once an e-waste hub emerges, it must have some advantage to continue existing. Some of these advantages include local expertise accumulation,
improved supply chains, and improved economies of scale \cite[Sec. 3.5]{davisPollutionHavenHypothesis2019}.
A good example of this is Lagos, Nigeria. The Nigerian government decided to capitalize on their supply of local and imported e-waste.
Many Nigerians became experts in e-waste recycling and vocational schools, governments, and companies started training in it \cite[pp. 93-94]{sullivanTrashTreasureGlobal2014}.

\section{Problems with E-Waste} \label{sec:problems}

This section will enumerate some findings of the health, environmental, and equity effects of informal e-waste recycling,
the dominant form of e-waste recycling in the world \cite[pp. 10, 30]{baldeGlobalEwasteMonitor2024}.

\subsection{Health Problems}

Informally recycling e-waste can pose problems for human health. An example is found in \cite{xuDifferentialProteomicExpression2016}.
Elevated levels of cadmium (Cd) and lead (Pb) were found in the placentas of pregnant women who lived
near the Guiyu WEEE recycling site in China, compared to nearby Shantou \cite[Sec. 3.1]{xuDifferentialProteomicExpression2016}.
They found that the Guiyu women took longer to gestate their babies and the babies they gave birth to were smaller \cite[Sec. 3.1]{xuDifferentialProteomicExpression2016}.
They also note that this exposure causes significant changes in multiple proteins and that cadmium exposure, in particular, can slow fetal growth 
\cite[Sec. 5]{xuDifferentialProteomicExpression2016}.

For another Guiyu example, \cite{xuBloodConcentrationsLead2018} found that school-age children had high levels of mercury (Hg) in their blood.
The researchers note that exposure to lead and mercury in youth can cause damage to DNA, potentially increasing risk for cancers \cite[p. 1491]{xuBloodConcentrationsLead2018}.

In Vietnam, \cite{eguchiOccurrencePerchlorateThiocyanate2014} found high levels of perchlorate in the blood of people near a WEEE recycling facility,
in comparison with a rural site. Perchlorate can be used in explosive devices \cite[p. 29]{eguchiOccurrencePerchlorateThiocyanate2014}.
These high levels were not otherwise associated with dietary and lifestyle habits \cite[p. 33]{eguchiOccurrencePerchlorateThiocyanate2014}.

Sufficiently informal and unregulated practices put workers at increased risk of physical injury as well.
In Ghana, researchers studying the Agbogbloshie WEEE recycling site found that only 38\% of workers used gloves, goggles,
or other protective equipment \cite[Fig. 5]{adanuChallengesAdoptingSustainable2020}.
The same study found that only 23\% of surveyed workers “occasionally” used specialized cable stripping
machines instead of burning cables to extract the valuable materials within \cite[p. 4]{adanuChallengesAdoptingSustainable2020}.

E-waste contains many toxic and dangerous chemicals, according to a meta-analysis \cite{parvezHealthConsequencesExposure2021}.
Some of these chemicals include mercury, manganese (Mg), chromium (Cr), nickel (Ni),
polycyclic aromatic hydrocarbons, polybrominated diphenyl ethers, polychlorinated bisphenols, dechlorane plus, and many others
\cite[Sec. "Results"]{parvezHealthConsequencesExposure2021}.

\subsection{Environmental Problems}

Informally recycling WEEE can prove to be a serious environmental hazard. However,
it is sometimes viewed more as an economical concern, than an ecological one \cite[p. 133]{albuquerqueElectronicJunkBest2019}.
In this section, I will enumerate several findings from researchers.

Taking data from \cite[Tbl. 4]{dengAtmosphericLevelsCytotoxicity2006}, \cite[Sec. 6.7]{worldhealthorganizationregionalofficeforeuropeAirQualityGuidelines2000},
\cite[Tbl. 1]{varConcentrationTrendSeasonal2000}, and \cite[Tbl. 1]{kimChemicalCompositionFine2003},
I created a chart measuring the concentration of lead (Pb) in the air of Guiyu, China, in comparison to two other Asian
cities and World Health Organization estimates of non-urban areas in Europe.

\newpage

\begin{figure}[h]
    \centering
    \includegraphics[width=\textwidth]{"images/Lead Air.png"}
    \caption[Lead air concentration]{Lead air concentration in Guiyu vs. selected control areas \cite{dengAtmosphericLevelsCytotoxicity2006},
    \cite{worldhealthorganizationregionalofficeforeuropeAirQualityGuidelines2000},
    \cite{varConcentrationTrendSeasonal2000}, \cite{kimChemicalCompositionFine2003}}
\end{figure}

TSP refers to Total Suspended Particles (particles less than 30-60 micrometers) and
PM means Particulate Matter (particles smaller than 2.5 micrometers) \cite[p. 6945]{dengAtmosphericLevelsCytotoxicity2006}.
As one can see, both measures are far higher in Guiyu than in other Asian cities like Tokyo and Seoul.
They are also much higher than what the World Health Organization estimates the average to be for non-urban areas in Europe.
Please note that for \cite{kimChemicalCompositionFine2003}, I considered the “Fine” parameter in Table 1,
as that relates to the same PM measure of 2.5 micrometers \cite[p. 754]{kimChemicalCompositionFine2003}
that \cite{dengAtmosphericLevelsCytotoxicity2006} uses. Please note that \cite[Sec. 6.7]{worldhealthorganizationregionalofficeforeuropeAirQualityGuidelines2000}
lists their measurement as 0.15 micrograms per cubic meter, which is equal to 150 nanograms per cubic meter, as shown in the above chart.
Also note that the WHO did not specify if they considered TSP or PM measures
\cite[Sec. 6.7]{worldhealthorganizationregionalofficeforeuropeAirQualityGuidelines2000}.

In \cite{ouaboEcologicalRiskHuman2019}, researchers took soil samples from the Makea, Ngodi, and New Bell informal e-waste recycling sites in Douala, Cameroon.
They compared these to soil samples from a control area. They were measuring concentrations of the heavy metals lead (Pb), nickel (Ni), zinc (Zn),
cadmium (Cd), chromium (Cr), copper (Cu), and cobalt (Co).

\newpage

\begin{figure}[h]
    \centering
    \includegraphics[width=\textwidth]{"images/Soil Samples.png"}
    \caption[Soil Samples in Cameroon]{\cite[Fig. 2]{ouaboEcologicalRiskHuman2019}}
\end{figure}

The researchers found that the heavy metal concentrations in the three e-waste recycling sites were much higher
than in the control site. Of these, they noted that lead, nickel, copper, and cadmium posed the highest ecological risk.
This is particularly troubling for lead since it is much more concentrated in these samples than any other metal.
Chromium was considered lower, but still significant risk.  Zinc was considered the lowest risk, ecologically.
In all, all three sites were considered to pose very high ecological risk \cite[p. 7]{ouaboEcologicalRiskHuman2019}.

Researchers in \cite{sepulvedaReviewEnvironmentalFate2010} summarized the diverse ecological effects of informal
e-waste recycling in China and India. They produced the following diagram:

\newpage

\begin{figure}[h]
    \centering
    \includegraphics[width=\textwidth]{"images/Overview.png"}
    \caption[Ecological Problems Diagram]{Summary of ecological problems of informal e-waste recycling \cite[Fig. 1]{sepulvedaReviewEnvironmentalFate2010}}
\end{figure}

As one can see, informal e-waste recycling practices have diverse and wide-reaching environmental effects.
Each step in the informal process (dumping, dismantling, burning, and acid leaching) causes its own set
of environmental problems. Each of these effects are complexly related in a web.
To describe a few of these effects: leachates from bathing e-waste in acid can leak into the soil;
fumes from burning can be transported long distances; effluents from the byproducts of leaching
can get into water and the life-forms there; and coarse particles from shredding e-waste can enter
the soil and eventually the groundwater. All these effects can harm humans as well, as we breathe the
contaminated air; drink, bathe, and fish in the contaminated water; and eat contaminated animals and plants.

\subsection{Equity Problems}

As shown in \ref{sec:where}, WEEE is frequently sent from richer to poorer countries, sometimes illegally
\cite{unitednationsofficeondrugsandcrimeunodcTransnationalOrganizedCrime2013}. If we also consider that poorer countries are less likely
to have developed formal WEEE recycling practices and infrastructure \cite[p. 290]{perkinsEWasteGlobalHazard2014},
we can conclude that less developed countries take on a greater health and environmental burden
in processing e-waste than more developed countries. 

However, poorer countries can be active attractors of WEEE because it is a valuable income stream for the local economy 
\cite{davisPollutionHavenHypothesis2019}, \cite[Sec. 6]{sthiannopkaoHandlingEwasteDeveloped2013}.
Although some workers may owe their livelihoods to e-waste, and smaller economies may benefit from the informal industry, we can't
ignore the serious health and ecological problems these people are the primary victims of.

This poses an inherent contradiction. If the Global North were to stop exporting e-waste to the Global
South completely, this may cause wide-scale livelihood loss and denial of opportunity of the sort that
caused some informal e-waste recycling sites to emerge in the first place \cite[p. 43]{davisPollutionHavenHypothesis2019}.
But continuing to supply the Global South with such dangerous materials poses long-term ill-effects as well.
Maintaining e-waste hubs in the Global South for local economy and ensuring local health and environmental safety are at odds.

\section{What's Being Done?}

This section will examine what is being done about the problem of e-waste. It will focus on two scopes:
the international scope, and the Canadian scope. In Appendix B, an additional scope will be considered: the local, Montreal scope.

\subsection{Internationally}

Internationally, the main document one will discover in searching about e-waste is
The Basel Convention on the Control of Transboundary Movements of Hazardous Wastes and their Disposal,
colloquially known as the Basel Convention \cite{unitednationsBaselConventionControl1989}.
This is a UN agreement that was first adopted in 1989 and first came into force in 1992 \cite{PartiesBaselConvention}.
Canada was an early adopter of the Convention, signing the agreement in 1989 and ratifying it in 1992 \cite{PartiesBaselConvention}.
Notably, The United States signed the agreement in 1990, but never ratified it \cite{PartiesBaselConvention}.
The agreement outlines principles and standards for exporting and importing hazardous wastes.
However, e-waste is sometimes considered a loophole \cite{wangTakeResponsibilityElectronicwaste2016}.
As I mentioned in the introduction, used-EEE is a usefully different category than WEEE \cite[Sec. 1]{schluepWhereAreWEEE2012}.
This lets developed countries trade e-waste as used electronics, not hazardous and toxic waste, letting them skirt the Convention
\cite{wangTakeResponsibilityElectronicwaste2016}, \cite[p. 697]{ghoshWasteElectricalElectronic2016}.

Another potential criticism of the Basel Convention is that it is a top-down approach that is not as
efficient or effective as it could be because it does not consider the private sector and was not created
with enough scientific data \cite{awasthiCircularEconomyElectronic2019}.

An example of this is in India. Where, even though transboundary flows of WEEE are illegal,
it still happens in large volume and adds a valuable income stream to the local economy \cite[p. 697]{ghoshWasteElectricalElectronic2016}.
Another example is China. It signed the Basel Convention and banned the import of WEEE,
but material still finds its way in, smuggled via Vietnam \cite{ghoshWasteElectricalElectronic2016}, \cite[p. 15]{wangEWasteChinaCountry2013}.

In response to the Basel Convention, regional agreements were devised that extended the Convention and hoped to patch its oversights.
One such agreement is the Africa-centric Bamako Convention \cite{u.n.environmentBamakoConvention2017}.
The Basel Convention encourages regional agreements such as these \cite{u.n.environmentBamakoConvention2017},
\cite[Art. 11]{unitednationsBaselConventionControl1989}.

As mentioned in \ref{sec:where}, tracking the transboundary movements of e-waste poses a significant challenge.
This is an important area requiring improvement. If e-waste is to be better regulated worldwide,
it must be possible to map its flows (imports, exports, interregional, and intraregional), and in what quantity.

In the European Union, the Restriction of Hazardous Substances in Electrical and Electronic Equipment (RoHS)
first came into force in 2003. The second version came into force in 2011 \cite{RoHSDirectiveEuropean2022}.
This directive restricts the use of hazardous material in EEE and is concerned with the ecological recovery
of WEEE items \cite[Art. 1]{europeanunionDirective2011652011}. It requires that manufacturers ensure their
products do not contain more than the maximum allowable amount, by weight, of certain substances
\cite[Art. 4]{europeanunionDirective2011652011}. One problem with the RoHS is that EU member states
are able to decide upon penalties for non-conformity on their own \cite[Art. 23]{europeanunionDirective2011652011}.
Another problem is that the scope of the Directive is limited by the large number of product
categories that are exempt from it, like military equipment and implantable medical devices
\cite[Art. 2]{europeanunionDirective2011652011}.

\subsection{In Canada} \label{sec:incanada}

Canada exists in a formal WEEE-recycling context \cite[p. 157]{ceballosFormalElectronicRecycling2016}.
However, there exists no federal legislation concerning e-waste.
Instead, the individual provinces and territories decide their own rules \cite[p. 69]{baldeGlobalEwasteMonitor2024}.
Interestingly, the territory of Nunavut has no WEEE legislation at all \cite[p. 70]{baldeGlobalEwasteMonitor2024}.

The closest Nunavut has to this is a non-binding guideline on hazardous waste that only briefly
mentions one proper, non-battery, e-waste category: computer parts
\cite[Sec. 2.1]{governmentofnunavut-departmentofenvironmentEnvironmentalGuidelineGeneral2010}.
In terms of indigenous Canadian issues, over 80\% of Nunavut's population is Inuit \cite{governmentofcanadaCensusProfile20162017}.
An interesting avenue of further research is how the lack of WEEE regulation in Nunavut affects this population.

Of the provinces and territories that do have e-waste legislation, the Extended Producer Responsibility (EPR)
model is used as a basis for such laws \cite[p. 70]{baldeGlobalEwasteMonitor2024}.
The EPR model posits that it is producers (and by extension, manufacturers and first importers)
who bear the responsibility of managing their end-of-life products. Companies can pass this additional
cost on to consumers in the form of an environmental handling fee, charged at time of sale \cite{baldeGlobalEwasteMonitor2024},
\cite{environmentandclimatechangecanadaOverviewExtendedProducer2007}.

The EPR model can be divided into two subcategories: Individual Producer Responsibility (IPR)
and Collective Producer Responsibility (CPR) \cite[p. 1043]{atasuExtendedProducerResponsibility2012}.
In CPR, companies work together as a group to share the effort of responsibly recycling unwanted electronics.
A problem with this is that CPR forces sharing the costs associated with e-waste recycling, as well
\cite[p. 1043]{atasuExtendedProducerResponsibility2012}. This may discourage companies from
designing more recyclable products, as more recycling means more cost for them \cite[p. 1048]{atasuExtendedProducerResponsibility2012}.
The EPR model can also cause recycling to be emphasized over reuse \cite[p. 674]{dalhammarEnablingReuseExtended2021},
which I will later argue is a powerful way to reduce WEEE generation.

EPR has additional problems. One such problem is that it's difficult to define good targets for EPR effectiveness
\cite{huismanWhereDidWEEE2006}. Different materials, products, and processes all pose different logistical and
ecological problems, some of which are more important than others \cite[Tbl. 2]{huismanWhereDidWEEE2006}.
In some jurisdictions of Canada, such as Quebec, there are no penalties for enterprises failing
to disclose the effectiveness of their EPR efforts because such a mandate was repealed in 2023
\cite[Sec. 53.1 (6)]{quebecRegulationRespectingRecovery2011}. Also in Quebec, there are
no penalties for failing to set up drop-off centers as defined in Section 16, because such a penalty was also repealed in 2023 
\cite[Sec. 56.1 (2)]{quebecRegulationRespectingRecovery2011}. A third problem is that there may be no way to tell if
a Producer Responsibility Organization (PRO) (an organization set up to oversee the EPR efforts of its member company/ companies)
meets targets due to a lack of disclosure, as seen in Quebec with the EPRA \cite[p. 1036]{leclercInformalEWasteFlows2023a}.

Thankfully, some parts of Canada have landfill bans on e-waste, such as Newfoundland and Labrador
\cite{departmentofenvironmentandclimatechangepollutionpreventiondivisionLandfillBansSpecial2023}, Nova Scotia
\cite{departmentofenvironmentandclimatechangeMaterialsBannedDisposal2009}, Prince Edward Island
(see section “electronics considered for special disposal only”) \cite{princeedwardislandWasteWatch2016},
and Vancouver (see section “Banned product stewardship materials”) \cite{metrovancouverDisposalBanProgram}.

In Canada, there are a few organizations of note. One of them is Electronic Product Stewardship Canada (EPSC)
\cite{EPSCElectronicsProduct}. This is a non-profit industry consortium. They favor regulations that are
flexible and industry-led and they focus on making businesses more efficient \cite[p. 8]{xavierCircularEconomyEwaste2021}.
For a more governmental approach, there exists the Canadian Council of Ministers of the Environment (CCME)
\cite{CanadianCouncilMinisters}. This is an intergovernmental group of federal, provincial, and territorial ministers
that was founded in 2004. They favor the EPR model and try to focus on reduction of consumption \cite[p. 8]{xavierCircularEconomyEwaste2021}.
They created a set of principles called \textit{Canada-Wide Principles for Electronics Product Stewardship},
which tries to harmonize Canada-wide approaches and standards 
\cite{canadiancouncilofministersoftheenvironmentccmeCanadaWidePrinciplesElectronics2004}.

The third and most interesting of these organizations is the Electronic Products Recycling Association (EPRA)
\cite{EPRAARPEElectronic}. This is an industry-led not-for-profit group that operates regulated Canada-wide
e-waste recycling programs. They are a Producer Responsibility Organization that uses the CPR model as multiple
companies are members of it. There is no obligation for companies or individuals to use the EPRA
\cite{leclercExtendedProducerResponsibility2020}. The EPRA, despite regulating e-waste
recycling centers nationwide, has optional membership \cite[p.294]{coteWorkerHealthFormal2021}.

Because of the fact that Canada has no federal WEEE legislation, companies that do business in
more than one province or territory, may find e-waste compliance difficulties between them
\cite[p. 70]{baldeGlobalEwasteMonitor2024}.

As a whole, Canada puts no pressure on manufacturers to encourage more recyclable or repairable electronics
\cite[p. 70]{baldeGlobalEwasteMonitor2024}. This is a point I will bring up later. 

In 2022, Canada generated approximately 774 million kilograms of e-waste
and formally recycled only around 89 million kilograms of it \cite[Tbl. A2.4]{baldeGlobalEwasteMonitor2024}.
This is a rate of about 11.5\%. When compared to the global formal e-waste recycling average of around 22.3\%
\cite[Fig. 7]{baldeGlobalEwasteMonitor2024}, we see that Canada does about twice as poorly as the already
low estimated global average. Canada does not seem to be actively trying to improve this statistic either,
as it has no recycling target in place \cite[Tbl. A2.4]{baldeGlobalEwasteMonitor2024}. However, 
\cite{xavierCircularEconomyEwaste2021} notes some challenges Canada faces when it comes to managing e-waste:
its large surface area, population density, and WEEE generated per capita. Canada has a large amount of WEEE
generated per capita, but its population density is sparse over its huge surface area.
This poses serious logistical and cost difficulties of properly and uniformly managing e-waste in Canada
\cite[Sec. 3.1]{xavierCircularEconomyEwaste2021}.

\section{Solutions}

Now that the serious problems of e-waste and what is being done about them have been examined,
I will propose some solutions. This will include what people can do as individuals,
what Canada can do as a country, and what all countries can do together. 

\subsection{Solutions for Individuals}

This section is focused on what people as individuals can do to reduce the problem of e-waste.

\newpage

\subsubsection{The 3R Initiative}

\begin{wrapfigure}{l}{0.25\textwidth}
    \centering
    \includegraphics[width=1in]{images/3R.png}
    \caption[The Current 3R]{Reduce, Reuse, Recycle}
\end{wrapfigure}

The 3R Initiative was first introduced by Japan in 2004 \cite{3RInitiative}, \cite{sthiannopkaoHandlingEwasteDeveloped2013}.
This is colloquially referred to as “Reduce, Reuse, Recycle”. The order of these activities matters.
First, a person should reduce their consumption of, in this case, electronic and electrical equipment (EEE).
This means buying less of it, but it also means using it for as long as possible.
If one were to use their smartphones, for instance, for as long as they still functioned,
then this would put downward pressure on the number of new smartphones purchased.
This would also put downward pressure on the number of smartphones discarded per given time frame.
Regular maintenance, both physical and digital, can keep devices working longer.

Secondly, after a reduction of consumption has been achieved, a person should reuse what EEE they can.
It's always possible to donate a working EEE item to friends, family, or thrift stores.
Even if a device no longer works well, it may be possible to give it new life by reconsidering the
context in which it's used. For instance, an old laptop that no longer functions to modern standards
can be refurbished into a simple home server by installing a lightweight operating system onto
a fast new hard drive. This assumes that the user is tech-savvy enough to do this, but simpler
options are possible too. Old MP3 players can be displayed artistically, and old landline phones
may be able to function as walkie-talkies if that functionality is available. It's even possible
to turn old junk into household sculptures, such as in \cite{deroseProjectsEWaste}. Finally,
after reducing consumption of EEE and reusing as much of it as can be done, recycle it formally,
if possible. This makes it more likely that the material will be disposed of correctly.

\subsubsection{A Better Version of 3R}

The above model may be compelling. It tries to address the main problem of e-waste
(how much is sent to recycling and potentially exported to the Global South) in a simple,
effective way. But it is possible to go further. The 3R Initiative has existed for 20 years
already, and e-waste generation is only increasing. The model considers the proper order of
reduction, reuse, and recycling, but does not intrinsically consider the proper proportion of
these activities. Each of the three “R”s is given equal importance. A much more effective
strategy at reducing e-waste sent to less developed countries is one that heavily emphasizes
reduction of consumption and heavily de-emphasizes recycling.

\begin{figure}[h]
    \centering
    \includegraphics[width=\textwidth]{"images/Triangle ver3.png"}
    \caption[A Better 3R]{Reduce, Reuse, Recycle: revamped. Colorblind-friendly version on right}
\end{figure}

This is a simple diagram showing an improved version of the 3R model.
It maintains the order of the previous model through the triangular shape,
and reinforces through color and proportion that the best and most important of the three
“R”s is reduction, next is reuse, and as a last resort, recycling.

As I have shown in \ref{sec:problems}, informal recycling is quite dangerous,
so reducing the material that Canadians recycle reduces the material that is
potentially exported for informal recycling in the Global South.
I have also shown that formal recycling, though better regulated,
may still have serious problems and oversights a lack of research hasn't found yet.
This modified 3R approach, in which the order and proportion of the “R”s is more obvious,
may be more powerful than the simple 3R model of today.

\subsection{Solutions for Canada}

One of the biggest problems Canada faces regarding e-waste, is that is has no federal e-waste laws.
One of the most obvious solutions, then, is to federalize e-waste legislation. This legislation should,
of course, concern itself with recycling, but also collection, landfilling, manufacturing,
and other critical facets of electronic devices and their waste.

If this is impossible, likely due to lack of political will, then a second-best solution
would be for the CCME to be more aggressive in their approach.
Their goal is to harmonize provincial and territorial approaches to e-waste, among other things,
yet Nunavut still has no such laws, and each province still has conflicting standards and practices.
It may be possible that the CCME, given more power and oversight, could accomplish much more.

If the CCME truly isn't the correct organization for the job, and the federal government
is deeply against regulating e-waste itself, then perhaps Ottawa could create a new, Canada-wide,
organization dedicated to e-waste regulation. This organization would need to have significant
oversight and authority. But, just as important, it should have relevant researchers in key positions0.
Either way, private sector, membership-optional Producer Responsibility Organizations like the
EPRA are not as effective as we need. This is obvious considering that Canada has half the
already low global rate of formal e-waste recycling and no target in place to get better
\cite[Tbl. A2.4]{baldeGlobalEwasteMonitor2024}.

Canada should pressure manufacturers to produce more repairable items that are made from fewer
harmful chemicals if they are to be sold on the Canadian market. It is obvious that,
in North America, The United States is the much bigger market, so it possesses much
greater negotiating power. It is unrealistic to expect manufacturers to produce separate
product designs for the much smaller Canadian market. Therefore, Canada should at least
try to collaborate with the US to create economic pressure on technology manufacturers.
The North American market is much more valuable than just the Canadian market and manufacturers
are unlikely to want to give up access to it. This is why working with the US is crucial for
the success of this idea. Focusing on making products with less hazardous material also
espoused by \cite[Fig. 2]{ceballosFormalElectronicRecycling2016}.

In terms of repairability, Canadians should have the right to repair their electronic devices.
This goes well with the above point of favoring more repairable product designs.
In the Canadian Senate, Bill C-244 \cite{miaoActAmendCopyright2021} seeks to amend the Canadian Copyright Act.
This amendment would allow Canadians the right to diagnose, maintain, and repair certain copyrighted
works even if it means breaking protections they have in place, so long as doing so does not
otherwise breach copyright. This bill, while a good idea, may be too limited in scope,
primarily concerning itself with computer programs and audio recordings.
A stronger version of this amendment was proposed in Bill C-272 \cite{mayActAmendCopyright2020}, 
which has not made it past the House of Commons of a previous Parliamentary session.
A recent bill that directly mentions the phrase “right to repair” is Bill C-273,
introduced in the 40th Parliament \cite{masseActAmendCompetition2009}.
It is primarily concerned with cars, but has the interesting stipulation that manufacturers
be required to provide vehicle owners and garages the necessary tools and training information
to diagnose and repair vehicles on their own. The Bill did not make it past the House of Commons
and was rejected in November 2009 \cite{masseActAmendCompetition2009}.

Canada should implement more e-waste landfill bans. As an aspirational goal, Canada should ban
e-waste in landfills nationwide. Canada should take more pains to formally recycle what e-waste
we generate and export less of it to the Global South, where it is typically recycled informally.
However, some informal flows are desirable, such as donation. If Canada were to ban WEEE from
landfills and export less of it, that doesn't mean that all e-waste in Canada should be formally
recycled. It's good to feed into the circular economy and to find new contexts for old items,
when possible. As I have shown earlier, formal recycling may still be problematic and its
ecological and health effects are under-studied.

\subsection{Solutions for the World}

Globally speaking, there should be more regional e-waste agreements like Africa's Bamako Convention
\cite{u.n.environmentBamakoConvention2017} and Europe's RoHS Directive \cite{RoHSDirectiveEuropean2022},
\cite{europeanunionDirective2011652011}. The Basel Convention may be too top-down to be as effective
as necessary, as argued in \cite{awasthiCircularEconomyElectronic2019}. Smaller, more regional
agreements may be able to bridge the gap between global ambition and regional reality.

Better tracking and reporting of e-waste flows is necessary. This requires each country's
transparency and honesty. If we can't have accurate information about how much e-waste
there is and where it's going, it will be much harder to solve the problem of dangerous
material ending up in locations where it can't be safely handled.

Illegal flows of WEEE and used-EEE need to be treated as seriously as other forms of illegal
contraband smuggling. This material causes real harm to people and the environment.
Governments should view it as being in their selfish interest to keep their people
and natural resources safe from smuggled, potentially toxic material.

\section{Conclusion}

The improper handling of e-waste causes serious ecological and health concerns.
In addition, because e-waste is so frequently sent to the Global South, which is
typically not as formally equipped, the brunt of these negative consequences is
suffered primarily by people of less developed countries. This raises serious equity concerns.

What Canadians should do to minimize our own contribution to these serious issues,
is to consume electronic devices less and focus on reusing what's possible.
This will cause less e-waste to be sent to other countries. Reduction of consumption
and reuse of old devices should take precedence over recycling.
This considers the unsafe nature of informal e-waste recycling,
and the potentially less-than-safe nature of formal e-waste recycling.

But this, too, causes a serious problem. As demonstrated previously,
countries in the Global South are not necessarily passive victims of the
trading of dangerous materials. They can be active attractors of global WEEE
because of its lucrative nature. If Canada were to just turn off the supply of
our WEEE to the world, that would reduce valuable income streams to countries
that depend on it for their economic growth \cite[p. 43]{davisPollutionHavenHypothesis2019}.

According to \cite[p. 26]{baldeGlobalEwasteMonitor2024}, the generation of e-waste is expected
to increase globally, at least until 2030. There's no reason to think that this trend won't
continue beyond 2030, as well. The electrification of the world has already brought serious
economic and lifestyle gains to people around the globe. More luxury and ease will always be
compelling, so it stands to reason that the increasing use of electricity and electronics
will continue indefinitely.

If we assume that the generation of e-waste will increase globally, at least for the foreseeable future,
then even if Canada reduces the proportion of e-waste that it exports,
it is possible that the absolute tonnage of it may still grow.

In addition, if the waste that Canada does export to the Global South were made of safer,
more repairable materials as I suggest, the long-term average safety of informal recycling
should improve over time.

This combined strategy of reducing Canada's contribution to the e-waste problem,
while maintaining income streams for the safer, cleaner informal recycling of it in
the Global South will produce an ideal outcome. Canada will Reduce, Reuse, and Recycle
in the correct order and proportion, and when Canada does export WEEE for recycling elsewhere,
it will be safer for the recipients. The receivers of this safer e-waste won't
even have had to do anything. This will allow them to formalize their own processes at their own pace,
without the threat of health or ecological emergency forcing them to hasty, expensive action.

\appendix

\newpage

\section{Canada and the Global E-Waste System}

\begin{figure}[h]
    \centering
    \includegraphics[width=\textwidth]{"images/WEEE-export-ver7.png"}
    \caption[Stock and Flow Diagram]{Stock and flow diagram for WEEE export, including Canadian targets}
\end{figure}

The above stock and flow diagram illustrates the global system of e-waste and Canada's place within it.
It shows both global and Canadian stocks of WEEE being generated from arbitrary sources (far left).
These items are then exported to developing countries, contributing to their internal supply of WEEE.
The amount of WEEE in a developing country affects the number of jobs to process it, and thus,
income available for WEEE workers in that country. These jobs are typically dangerous, as discussed earlier.
The stock of workers who process WEEE in a country affects how much is processed.
This is typically informal in developing countries. If Canadian exports of WEEE were influenced
by more stringent WEEE exporting targets, this will decrease the amount of WEEE-based income available
to workers in developing countries. But this will also reduce the negative ecological and
health effects thereof.

Both Canadian and global contributions to WEEE export are included in the diagram because it's
important to note that even if Canadian policies regarding WEEE export were stricter,
the activities of all other countries will be unaffected. Thus, the two belong on separate, parallel tracks.
	
Canada should want the best outcome possible: contributing less to global pollution and health hazards,
while still maintaining the livelihoods of those who depend on its WEEE.
This is a balance that needs consistent monitoring and honest reporting.
Due to the overall lack of WEEE export reporting and its frequently illegal nature, this is difficult.

Ideally, Canadian WEEE exporting targets should be informed by a weighted balance of its negative effects,
and its economic utility. This is denoted on the diagram by the loop on the bottom denoted “B”.
In these diagrams, “B” is used to signify a balancing loop \cite[p. 27]{donellah.meadowsThinkingSystemsPrimer2009}.
This loop is balanced because Canadian WEEE exporting targets effect jobs in the Global South and
their negative aftereffects. These factors, in turn, should affect Canadian WEEE exporting targets
in a considered, measured way.

Note that this diagram assumes that some of the stock of Canadian WEEE makes its way to developing countries.
This may be by exporting “used electronics”, as shown earlier, or indirectly via illicit means.
Due to the global lack of detailed, honest reporting, the exact tonnage that Canada exports to
the Global South remains unclear.

\section{What's Being Done in Montreal?}

In this appendix, I will review the current state if what is happening in Montreal regarding properly
handing e-waste. I will review public and private options for individuals to formally recycle
their electronics. I will then discuss broad trends in what Montrealers have been doing with
their electronics over a ten-year period.

\subsection{Quebec Provincial Law}

Quebec has participated in Extended Producer Responsibility since 2011 \cite{quebecRegulationRespectingRecovery2011}.
Our provincial law is called \textit{Règlement sur la récupération et la valorisation de produits par
les entreprises}, which in English, translates to \textit{Regulation respecting the recovery and
reclamation of products by enterprises} \cite{quebecRegulationRespectingRecovery2011}.
This is Regulation Q-2, r. 40.1 of the \textit{Loi sur la qualité de l'environnement}, or
the \textit{Environment Quality Act} \cite{quebecEnvironmentQualityAct1972}.

Typical WEEE products are defined in Chapter 4, Division 1 \cite{quebecRegulationRespectingRecovery2011}.
Note that, consistent with the assertion in \cite[p. 20]{baldeGlobalEwasteMonitor2024} 
that batteries are a separate waste stream, the Regulation defines batteries in Chapter 4, Division 2
\cite{quebecRegulationRespectingRecovery2011}.

As found in Section 6 \cite{quebecRegulationRespectingRecovery2011}, the Regulation gives enterprises
the discretion to join an Individual (IPR) or Collective (CPR) Producer Responsibility Organization.
The distinction between these was discussed in \ref{sec:incanada}.

\subsection{Private and Public E-Waste Recycling in Montreal}

Residents of the Agglomeration of Montreal can use one of Montréal's seven \textit{Ecocentres}
\cite{villedemontrealTakeItemsEcocentre}, \cite{villedemontrealTakingItemsMaterials}.
These are publicly accessible waste disposal facilities around the island of Montreal.
They accept both small devices like phones and computers, as well as large appliances like refrigerators.
However, there are serious limitations to their effectiveness.
Residents must bring their own materials, as there is no pick-up service available.
This means that people must haul their own heavy items like stoves.
A compounding problem to this is that residents must unload their own heavy materials.
An additional problem is that users must be residents of the Agglomeration of Montreal,
not all areas in the Greater Montreal Area are included, such as Laval. Users of an
\textit{Ecocentre} must bring photo ID and other accompanying documents, as well. Each \textit{Ecocentre}
also has a list of permitted vehicles.

For a private-sector option, \textit{Electrobac} is a company that accepts small e-waste items
in over 270 collection bins in Quebec and Ontario \cite{Electrobac}. They are EPRA certified
and accept a variety of small items like phones and cables. A problem with \textit{Electrobac}
is that their scope is small, only operating in two provinces and with a limited number
of collection bins to share among them. Also, they don't accept large items, as their
collection method is exclusively via their small bins.

For another private-sector option with larger scope, the Electronics Recycling Association (ERA)
\cite{electronic.recycling.associationComputerRecyclingVancouver} operates in Canada
and the United States. They offer a pick-up service and work with commercial and industrial
generators of e-waste as well as individuals. A downside to them is that they do not appear
to be EPRA certified. They do not advertise as such on their website and do not appear on the
EPRAs official list of Stewards \cite{electronicproductsrecyclingassociationepraStewardList}.

\subsection{The E-Waste Habits of Montrealers}

Over a 10-year period, from 2011 to 2021, data was collected (every two years) as part of the
Households and the Environment Survey \cite{statisticscanadaHouseholdsEnvironmentSurvey2022}.
This survey measures Canadians' environmental practices. From this survey,
two datasets were created that are of relevance to e-waste. The first and most important of these is
\cite{statisticscanadaTable3810015401Electronic2022}. It measures Canadians e-waste activities
regarding several product categories. The second dataset relates to household hazardous
waste, and is included for completeness only, as it contains similar statistics for Canadians'
activities regarding batteries \cite{statisticscanadaTable3810015501Household2022}.

The code I wrote to process these datasets into the charts I created is available
in this public, open-source repository: https://github.com/BlackSound1/SUST601-Project.
It is important to at least skim each Python notebook (the .ipynb files), as I explain exactly
how each dataset is processed and the decisions I made along the way.

In the analysis of these data, a chart is presented per product category.
It reveals the specific activities Montrealers took toward that specific type of
item once they had products they wished to dispose of. The chart is organized by year on the x-axis.
Colorblind friendliness was considered as much as possible.

Important to note: in each product category, not all possible e-waste activities have data available.
This is always represented with a grey 'No data' bar. This is computed by adding up the percentages
of all the other e-waste activities available in a given year, and if the result is lower than 100\%,
marking the difference as 'No data'. Another note: some bars clearly add up to more than 100\%.
This is possible because one of the e-waste activities is “Sill had unwanted \_\_\_\_\_ at the time of interview”.
It is entirely feasible, for example, for a person to have donated an e-waste computer and still
have an e-waste computer they haven't done anything with yet. This lets it be possible for an
interviewee to have selected more than one of the e-waste activities.

One serious downside to these datasets is that the proportion of Montrealers throwing an item in
the garbage is not available for any product category (except batteries, which are not e-waste).
The e-waste activities revealed in the following charts are either all positive activities like
donation or returning, or neutral activities like doing nothing. Therefore, their relative
proportions increasing or decreasing don't matter as much as how these activities would fare
against the unknown proportion of Montrealers disposing of their items in the garbage.
Since that perspective is not known, it is not possible to make conclusions as to the degree
of the environmental friendliness of Montrealers' e-waste activities.
This is why this analysis is relegated to an Appendix: it is included for discussion and comparison alone.

\subsubsection{Computers}

\begin{figure}[h]
    \centering
    \includegraphics[width=\textwidth]{"images/computers_stacked_plot.png"}
    \caption[Computers Chart]{What did Montrealers do with computers?
    Data from \cite{statisticscanadaTable3810015401Electronic2022}}
\end{figure}

One heartening conclusion we can infer from this chart is that the proportion of Montrealers
who took their e-waste computers to a depot or drop-off center increased over the study period.
This is a trend that will reveal itself in the other product categories.

The proportion of Montrealers who still possessed e-waste computers when questioned remained
relatively consistent over the study period. This may indicate that when Montrealers got new computers,
they didn't necessarily discard their old ones. As speculation,
it could be that the old computers were unwanted, but the owner felt that discarding
them could pose some risk of losing data.

Another point to discuss is Montrealers donation habits for e-waste computers.
Although 2015 has no available data for this, it seems like the trend was declining until 2015,
after which it started slowly expanding again. It's good that this activity has been increasing
in popularity since 2017, but it's unfortunate that it remains such a small proportion of all
e-waste activities for computers.

Based on the given data, Montrealers returned the greatest proportion of e-waste computers
to suppliers or retailers in 2017, with decreasing popularity after.

\newpage

\subsubsection{Printers}

\begin{figure}[h]
    \centering
    \includegraphics[width=\textwidth]{"images/printers_stacked_plot.png"}
    \caption[Printer Chart]{What did Montrealers do with printers?
    Data from \cite{statisticscanadaTable3810015401Electronic2022}}
\end{figure}

Once again, we see that the general trend of people taking their unwanted printers to a depot
or drop-off center is increasing. The proportion of Montrealers who still possessed unwanted
printers to dispose of also remains relatively constant, however.

For the other e-waste activities, the data is too sparse to conclude any specific trends.

\newpage

\subsubsection{Televisions}

\begin{figure}[h]
    \centering
    \includegraphics[width=\textwidth]{"images/televisions_stacked_plot.png"}
    \caption[TV Chart]{What did Montrealers do with TVs?
    Data from \cite{statisticscanadaTable3810015401Electronic2022}}
\end{figure}

For televisions, the drop-off center trend is slightly less obvious. The later years of the
study have a higher proportion of Montrealers disposing of their TVs at a drop-off center
than the earlier years. However, 2021 has a lower proportion than 2019. It remains to be seen
whether this proportion will increase again. However, it can be cautiously said that the fact
that the later years show this general increase over the earlier years is an encouraging fact, by itself.

Another interesting observation is that the proportion of Montrealers returning their TVs to a
supplier or retailer is consistently small. To speculate, people might feel that TVs are quick
to become obsolete so retailers might be unlikely to resell them.

The trend of Montrealers donating their e-waste TVs was decreasing until 2019, when there was no
usable data. It appears to have increased in 2021. I can't conclude whether this is truly good,
because if the 2019 data were available and it showed a large proportion, then the 2021 value
would actually be a decrease.

\newpage

\subsubsection{Cell Phones}

\begin{figure}[h]
    \centering
    \includegraphics[width=\textwidth]{"images/phones_stacked_plot.png"}
    \caption[Cellphone Chart]{What did Montrealers do with cell phones?
    Data from \cite{statisticscanadaTable3810015401Electronic2022}}
\end{figure}

Although there are dips in the proportion of Montrealers who took their e-waste cell
phones to a drop-off center, the overall trend is increasing.

Interestingly, the proportion of Montrealers who still owned unwanted cell phones to
dispose of has remained consistent for the last several years, since 2015. Again,
it might be the case that people fear disposing of their cell phones, as they can
contain so much personal data.

Montrealers taking their e-waste cell phones back to a supplier or retailer is an
overall decreasing trend since 2013, perhaps for the same reason.

\newpage

\subsubsection{Landline Phones}

\begin{figure}[h]
    \centering
    \includegraphics[width=\textwidth]{"images/landlines_stacked_plot.png"}
    \caption[Landline Chart]{What did Montrealers do with landline phones?
    Data from \cite{statisticscanadaTable3810015401Electronic2022}}
\end{figure}

This product category shows some of the limitations of this dataset. Only three years are available,
and only two e-waste activities. For the trend of Montrealers giving their old landlines to a drop-off center,
the data paints a less obvious picture. If one were to create a line of best fit between the blue bars
of 2019 and 2021, the line would be going up. But since there's so little available data, the final
year showing a downward trend is significant.

There appears to be no appreciable difference in the proportion of Montrealers who still had e-waste
landline phones. This may be because Montrealers are not buying new landlines, as they can be seen as
obsolete compared to smartphones. This would leave the number of unused landlines consistent if the
owners never decided to discard them.

\newpage

\subsubsection{Audiovisual Equipment}

\begin{figure}[h]
    \centering
    \includegraphics[width=\textwidth]{"images/av_stacked_plot.png"}
    \caption[AV Chart]{What did Montrealers do with AV equipment?
    Data from \cite{statisticscanadaTable3810015401Electronic2022}}
\end{figure}

This is the final normal WEEE product category examined here. I analyzed a few others in the
code written to explore this dataset, but those categories had data that were too sparse to use effectively.

The overall trend of Montrealers giving their e-waste to a drop-off center again appears to be rising,
even though this trend does have a noticeable dip in 2017.

Data on the other e-waste activities are too sparse to come to interesting conclusions.

\newpage

\subsubsection{Batteries}

This is a product category that is not normally considered e-waste, as shown earlier in the report.
However, for comparison, I will still briefly discuss findings on Montrealers habits toward them.

\begin{figure}[h]
    \centering
    \includegraphics[width=\textwidth]{"images/batteries_stacked_plot.png"}
    \caption[Batteries Chart]{What did Montrealers do with batteries?
    Data from \cite{statisticscanadaTable3810015401Electronic2022}}
\end{figure}

The proportion of Montrealers taking their dead batteries to a drop-off center seems consistent
throughout the study period, as does the proportion of Montrealers who still had unwanted batteries.
A possible reason for this may be people waiting until they have enough dead batteries to be worth
properly disposing of, which may take a long time.

Thankfully, the proportion of Montrealers putting their dead batteries in the garbage is quite
small compared to the other activities. Note that this is the only category with such data.

\printbibliography[title={Bibliography}]

\end{document}
